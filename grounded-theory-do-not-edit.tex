% Sample apostrophy`s to remove team`s 

\chapter{Grounded Theory}
\section{GroundedTheory}

Constructivist Grounded Theory \cite{Charmaz} provides an iterative approach to data collection, data coding, and analysis resulting in an emergent theory. 

Grounded Theory immerses the researcher within the context of the research subject from the point of view of the participants. As the research progresses, Grounded Theory allows the researcher to \quotes{incrementally direct the data collection and theoretical ideas.} The theory provides a starting place for inquiry, not a specific goal known at the beginning of the research. As the researcher interacts with the data, the data influence how the research progresses and directs the research direction. When starting a Grounded Theory research study, the core question is, \quotes{What is happening here?} \cite{GlaserTheoreticalSensitivity}. This can be applied to a domain of research, for example, \quotes{What is happening at Pivotal when it comes to software development?}

Charmaz encourages the researcher to \quotes{follow this grounded theory strategy: seek data, describe observed events, answer fundamental questions about what is happening, and then develop theoretical categories to understand it.}

Research questions should dictate which research methods are used. Charmaz argues that interviews are the best kind of data collection mechanism for certain kinds of research questions. (Todd: Need more detail)

Grounded Theory allows new aspects of the research to be added while gathering data and can even happen late in the analysis. The data collection process is guided by the research and altered as knowledge is accumulated. It is common for early research to illuminate new angles.``Grounded theory can give you flexible guidelines rather than rigid prescriptions." The theory provides a starting place for inquiry, not a specific goal known at the beginning of the research. The researcher`s guiding interests can be used ``as points of  ``departure to form interview questions, to look at data, to listen to interviewees, and to think analytically about the data."

There are three different versions of grounded theory: Classical grounded theory as promoted by Barney Glaser, Strauss and Corbin promoted by Anselm Strauss, and constructivist grounded theory promoted by Kathy Charmaz. Barney Glase and Anselm Strauss invented Grounded Theory while working on the \underline{Awareness of Dying} book \cite{GlaserAwarenessOfDying}. The success of the paper in unearthing novel results caused people to ask for more details about their process. In 1967, Glaser and Strauss wrote the book \underline{The Discovery of Grounded Theory} \cite{GlaserDiscovery}. In 1988, Strauss and Juliet Corbin published \underline{Basics of Qualitative Research}. Glaser strongly felt that the book mischaracterized the approach even asking Strauss to withdraw it from publication. Apparently both authors had different points of view for what they invented. In a heated response, Glaser published \underline{Basic of Grounded Theory Analysis} in 1992 clearly differentiating his perspective from Strauss. In time, Glaser’s approach became known as \quotes{Classical Grounded Theory} and Strauss’s approach named \quotes{Strauss and Cobin.} Kathy Charmaz was a PhD student of Glaser’s and relied upon Classical Grounded Theory for her career. In time, several reseacher’s approach evolved using variations (such as recording an interview) that Glaser clearly describes as not part of Classical Grounded Theory. Furthermore, they switched from a positivist point of view to a constructivist point of view. In XXXX, Charmaz published the book Constructing Grounded Theory. 

Classic Grounded Theory emphasises action and process. It looks at the question, ``What is going on here?" and decomposes it into ``What are the basic social processes?" and ``What are the basic social psychological processes?" 

The most common technique is interviewing. Interviews are coded and analyzed using constant comparison for the purpose of generating insights and recording memos. Creating memos is the chief activity and trumps all other activities. Memos are sorted into a paper draft with data collection and analysis continuing until theoretical saturation occurs. If a new avenues of reseach arises, sometimes additional data collection techniques are useful for generating the needed data. 
\subsection{Interview Technique}
Interviews are the most common technique of gathering data in a grounded theory study. Grounded Theory interviews are open-ended explorations of the participant’s perspective. Ideally, the interviewed does not force initial topics but merely follows the path of the interviewee. In Grounded Theory, interviewing is ``open-ended yet directed, shaped yet emergent, and paced yet unrestricted."

The interview initially starts with broad open ended questions to allow the emergence of unexpected narratives. 

Charmaz suggests \quotes{intensive interviews,} which are \quotes{open-ended yet directed, shaped yet emergent, and paced yet unrestricted} \cite{Charmaz}. Open-ended questions enter into the participant`s personal perspective within the context of the research question. The interviewer attempts to abandon assumptions to better understand and explore the interviewee`s perspective. Charmaz \cite{Charmaz} contrasts intensive interviews with informational interviews (collecting facts), and investigative interviews (exposing hidden intentions, practices or policies).

Charmaz describes Glaser and Strauss`s approach as \cite{GlaserDiscovery} `smash-and-grab` by prioritizing the needs of the researcher over the needs of the participants. Instead, she allows the participant to set the tone and pacing with the researcher matching and mirroring. 

The goal is for the researcher to understand the participant`s views and actions. The researcher does not need to agree with those views and actions, just interpret them. The researcher can discover what the participants take for granted and chooses do not share. 

The constructivist explores areas of theoretical interest when the participants mentions them. Opening questions slowly guide the conversation towards an area of interest. After understanding what the participants see, the researcher can guide them to provide more detail.

The constructionist endeavors to learn the meanings behind the participants word or phrases. The constructionist tries aims to not bring preconceptions to the research topic and understand the participants world view. 

The interviewer needs to balance the needs to enter into the interviewee`s narrative and getting enough detail to understand the ``analytic properties" Charmaz argues against placing time limits on interviews. Sometimes it is hard for the interviewer to balance listening to the participant while exploring topics of interest. For example, I thought that one of my interviews was wasteful since the interviewee spent 16 minutes answering one single question. Several times during the interviewee`s answer, I was tempted to interrupt. During data analysis, however, this segment turned into a treasure trove. Anxiety might have lead to prematurely moving on and missing interesting data. 

For organizational studies, Charmaz leads in with questions about collective practices and follows up with questions about the interviewee`s participation and point of view. For emotionally charged topics, an interview should proceed with safer questions before diving into personally challenging ones. 

When dealing with hard to discuss topics, Charmaz suggests asking the participant \quotes{
could I ask you about \underline{\hspace{0.8cm}} ?} instead of being direct with a question such as \quotes{tell me about \underline{\hspace{0.8cm}}?}.  This reduces the amount of control the interviewer has in the interview by empowering the interviewee.

When ending an interview, Charmaz considers the question \quotes{Is there something?} (preferred) instead of a traditional \quotes{is there anything?}. The former assumes that there is something that the person should reveal. The later tends to signal that the interview is ending and tends to close the conversation. I tried the variant, \quotes{Tell me one more thing} and it worked well for me.

As the research progresses, Grounded Theory allows the researcher to incrementally direct the data collection and theoretical ideas.

\textbf{Interview Guides:} Charmaz advocates that novices rely on interview guides. The guides should not be seen as scripts. The interviewer needs to be present with the interviewee, listening to both verbal and non-verbal communication. There are two benefits of interview guides: a) causes the researcher to reflect on sequencing and building of questions, b) reduces the chance of asking leading questions. First, the interview guide serves as a forcing function for the researcher to consider what data they want to collect and the best questions to guide the participant towards those topics. It enables the researcher to consider how and when to ask difficult questions. Second, relying on an interview guide will decrease the chances of a novice researcher blurting out an inappropriate question, in a moment of panic, or from accidentally asking a leading question. Experts tend to internalize the interview guide and follow where the conversion is leading them. The interview guide helps the researcher remove their own preconceived perspective which might taint the questions. Reflecting on the interview guide helps the researcher accomplish their research objectives. 

While grounded theorists agree that we shouldn`t preconceive the data or the analysis, grounded theorists disagree on what techniques constitute ``forcing" the data. Glasser \cite{GlaserIssues} argues against rules for proper memoing, interview guides, and units for data collection, whereas Charmaz argues that an open-ended interview guide for a semi-structured interview would help novices to not accidentally ask loaded questions. 

In review, Charmaz describes an ``intensive interview" process. The process involves open-ended questions. The purpose is for the researcher to enter into the participant`s personal perspective within the context of the research question. The interviewer needs to abandon assumptions and their own personal presumptive to understand and explore that of the interviewee. Charmaz contrasts intensive interviews from informational interviews which endeavor to collect accurate `facts` and investigative interviews that attempt to reveal hidden intentions or  expose practices and policies. The participant sets the tone and pacing with the researcher matching. 
\subsubsection{Interviewing Challenges}
There are potential barriers such as power imbalances, social norms, and prior knowledge, and forcing the data, that may affect the interview process and may affect what participants reveal. 

\textbf{The relationship of the researcher and the participant}, called \quotes{identity} by Charmaz, may affect the interview process. For example, power imbalances of a manager and employee or a professor and students.  Different kinds of strategies can be employed. A researcher could distancing oneself from positions of authority: a professor could interview students about learning techniques may decide to interview students in a different degree program. To alter power imbalance, a domain expert researcher might offer ``personal and professional views to encourage reciprocity." 

\textbf{Social norms and customs}, called \quotes{etiquette} by Charmaz. Participants might not want to reveal information to a stranger. A company rule of secrecy might make it difficult for a researcher to get the needed information. Framing questions is one way to elicit this kind of information. ``Some people have mentioned having negative pair programming sessions. Has that happened to you?" 

\textbf{Prior knowledge} can aid or create difficulties for the researcher. A researcher with domain expertise can know about interesting research questions and how to acquire certain kinds of data. That same domain expertise could blind the researcher to possible explanations that an outsider might see.

\textbf{Preconceived ideas} can be used as starting points for open-ended research. This enables the researcher to select an area of passion as long as they are guided by the data. The key is not to force preconceived ideas onto the data. The researcher can initiate with an interesting idea, yet it is critical for the researcher to be flexible with the research topic, listen to the ideas of the participants, and discover the emergent grounded theory. 

\textbf{Inaccurate interviews} It is possible for an interviewee to accidentally or intentionally fabricate their experience. In this case, the researcher may be entering an implicit collusion with their participants \cite{Yanos2008CollusiveObjectification}. Instead of treating such interviews as valueless, Grounded Theorists would examine the interchange trying to understand what is happening which may reveal information about forbidden topics and vulnerabilities of both parties. After-the-fact analysis may help the researcher from repeating such a performance by understanding how the participant was redirecting the interview. 

\sout{Ezzy (2010) says ``Textbooks on qualitative interviews are replete with masculine metaphors of conquest: probing, directing, questioning, active listening."}
\subsection{Field Notes for Participant Observation or Ethnographic Studies}
Complementary to interviewing, the researcher can record observations, either as a participant or as an ethnographer.

Grounded theory helps with participant observation and ethnographic studies. ``Grounded theorists select the scenes they observe and direct their gaze within them. Their field-notes show the actions, processes, and events that constitute what is happening in the setting. Grounded theory methods provide systematic guidelines for probing beneath the surface and digging into the scene. These methods help in maintaining control over the research process because they assist the ethnographer in focusing, structuring, and organizing it." \cite{Charmaz}

A Grounded Theory ethnographer can use recursive interviewing technique where each participants is interviewed multiple times. Each subsequent interview resumes the former conversation and deepens the relationship between the interviewer and the participant. 

Asking about specific terms can reveal internal concerns. For example, even with 20 years of software development experience, I had never heard the phrase ``bike-shedding" until a project where several engineers were using it to clearly communicate something succinctly. The phrase \quotes{bike-shedding} illustrated that the developers were worried about arguing over trivial matters because the team did not understand the deeper issues. Developers new to a programming language or technology might argue over trivial matters simply because they do not understand which arguments are worth having. Likewise, they might accept solutions because they do not understand the implications.

For an ethnographer, the \quotes{explicit integration of observation and interviewing affords immediate materials for analysis.} Grounded Theory adds checks and rigor to the ethnographer`s data collection and analysis. page 43 has a fantastic list of questions.

Collecting and analyzing field notes from participant observation is not easy. At the beginning of her research, Charmaz pictured a research experience where she would observe her research participants during the day and disappear into the coffee room to take detailed notes and write up her experience \cite{Charmaz}. She discovered that this is challenging in practice, as it can be difficult to disappear while observing activities. When the researched participant activity is intensive, such as the case with pair-programming, recording observations tricky. I relied on small notes on post-it notes (which were culturally acceptable compared to typing on a laptop) and detailed observations after work.

Field notes might
\begin{itemize}
\item Record individual and collective actions 
\item Contain full, detailed notes with anecdotes and observations 
\item Emphasize significant processes occurring in the setting 
\item Address what participants define as interesting and/ or problematic 
\item Attend to participants` language use 
\item Place actors and actions in scenes and contexts 
\item Become progressively focused on key analytic ideas.
\end{itemize}

It is possible for the researcher to oscillate between ``seeing data everywhere and nowhere, gathering everything and nothing." \cite{Charmaz}. When this occurs, the researcher returns to the iterative process of collecting, coding, analysing their data through constant comparison for guidance.

\subsection{Documents and other sources of data}
Charmaz suggests that researchers can undervalue documents and encourages researchers to view them as text assets that can be mined in the same way as interview notes or field observations. Since the researchers didn`t create the document, they can be seen as more \quotes{objective.} 

The researcher creates elicited documents as the researcher asks participants to answer questions, complete a survey, or write an account of their experiences. Extant documents exists without the researcher`s involvement. For extant documents, the researcher needs to understand the context in which the document is situated. 

Lindsay Prior argues that more than another voice, documents can do things that a participant can not. Beyond ``what do they contain?" we can ask the questions, ``what does the document do?", ``why was it created?", ``how was it created?", ``how are the documents used?, ``how do people interpret the document?" and ``what is not included in the document?" \cite{Prior2003UsingDocuments}
\subsection{Coding}
The purpose of coding is to fracture the source data into pieces that can be compared against each other during the constant comparison activity. Coding begins the framing from which the researcher builds analysis. In Constructivist Grounded Theory, interviews are recorded, transcribed, and then coded.

Charmaz advises to code everything during the early stages of research and see what emerges. During the initial phase, the researcher should be open to any direction. Once emergent themes arrive, then coding in later parts of the research can be focused around the themes. The focused coding phase allows the researcher to ``sort, synthesize, integrate, and organize large amounts of data."

There is no perfect coding scheme. Charmaz prefers one that is simple, short, direct, advice, analytic and spontaneous. Coding will be influenced by both the researcher`s and the participants` vocabulary. 

There are different styles of coding. Starting with Glasser (1978) some grounded theorists argue for using gerunds (e.g. dealing with uncertainty, exploring solutions to a problem) instead of topic or noun-based coding schemes (e.g. uncertainty, solutions). Relying on gerunds helps encourage the researcher to dig into the data and see the relationship between the participant and their actions. Charmaz strongly argues against labeling events, experiences, or topics as codes as the researcher gains little insight into the participant`s meaning, action, or point of view. Because Strauss and Corbin (1990, 1998) places less emphasis on this distinction, many grounded theorists take a more open stance to coding than relying on gerunds.

There are different granularity of coding. Charmaz prefers line-by-line coding and suggests there are times when even word-by-word coding is helpful. The line-by-line coding helps the researcher slow down and examine for nuanced interactions in the data. By 1992, Glasser prefers topic-by-topic or incident-to-incident coding. He advises against decomposing a single indecent. He feels that line-by-line coding produces too many codes, categories, and properties without producing analysis. While line-by-line coding does generate more codes, I appreciated the intimacy and thoroughness of line-by-line coding.

Charmaz suggests these heuristics to guide the researcher in coding:
\begin{itemize}
\item Remain open
\item Stay close to the data
\item Keep your codes simple and precise
\item Construct short codes
\item Preserve actions
\item Compare data with data
\item Move quickly through the data
\item \end{itemize}

TODD: where is this list from?
\begin{itemize}
\item Breaking the data up into their component parts or properties
\item Defining the actions on which they rest
\item Looking for tacit assumptions
\item Explicating implicit actions and meanings
\item Crystallizing the significance of the points
\item Comparing data with data
\item Identifying gaps in the data
\end{itemize}

At the researcher codes the data, Charmaz encourages the researcher to answer these questions:
\begin{itemize}
\item What process(es) is at issue here? How can I define it?
\item How does this process develop?
\item How does the research participant(s) act while involved in this process?
\item What does the research participant(s) profess to think and feel while involved in this process? What might his or her observed behavior indicate?
\item When, why, and how does the process change?
\item What are the consequences of the process?
\end{itemize}

Whenever insights arise, the researcher immediately stops the coding process and records the insight as a memo.

If for some reason, the codes remain mundane, Charmaz suggests ``coding the codes." In examining the codes, if there a larger process or activity? Perhaps there are patterns in the codes. 

For the constructivist, the researcher creates the code when the researcher examines the data and finds meaning within it. ``We construct" our codes because we are actively naming data." ``We choose the words that constitute our codes. Thus we define what we see as significant in the data and describe what we think is happening."

When observing routine activity, it can be challenging to see anything meaningful in the data. In this situation, the researcher can compare events to events looking for similarities and differences. 

Tensions may emerge in the coding. Tensions should be embraced rather than avoided or hidden. 

"Grounded theorists aim to code for possibilities suggested \textit{by} the data rather than ensuring complete accuracy \textit{of} the data." This stance provides opportunities for checking envisioned ideas with other data. Constant comparison allows the researcher to invalidate conjecture, an errant code will be detected and unsupported by other data samples. Ideas reflected in the data must earn their way into subsequent analysis. Constructivist grounded theorists acknowledge that coding will be influenced by both the researcher`s and the participants` vocabulary.

There are two phases of coding: initial coding and focus coding. 

\subsubsection{Initial Coding}
The initial coding process is where the researcher reads through data such as an interview transcript and marks fragments of data with a summary description or ``interpretive rendering".  The initial coding names every part of the data.  Initial coding is a slowing down process where the researcher becomes intimate with the data. The researcher can examine the data looking for implied things. While the emphasis is on summarizing the data and not analyzing it, the research should record any analysis ideas that could be explored later as memos. The researcher tries to avoid asking what the data means at this stage. Both actions and processes can be coded. The researcher does not want to bring preconceived ideas to coding.

After starting initial coding, constant comparison begins to help shape the direction of the research. 
\subsection{Focused Coding}
Once core categories emerge from constant comparison, the researcher shifts from initial coding to focused coding where the emphasis is flushing out the core categories and their properties. Focused coding continues until theoretical saturation occurs.
\subsection{Constant Comparison}

Constant comparison allows the researcher to identify \quotes{the conceptual relationship between categories and their properties as they emerged} \cite{GlaserBasics}, leading to an emergent theory.

In constant comparison, the researcher examines and compares codes to each other. It may be the case that two codes should be combined since they are describing the same phenomena. Codes that are related form a category. The researcher compares category to category looking for the relationship between them. The researcher periodically audits each category for cohesion by comparing its codes and moves codes that belong to a different category. Constant comparison compares incident to incident looking for patterns. Similarities strengthen the category while differences refine properties of the category. 

As categories emerge from the data, the researcher is looking for the core category as it defines the chief concern of the participants. Once the core category emerges, the researcher continues to collect and analyze data by examining the properties of the core category and its relationship to other categories until theoretical saturation occurs.

The researcher pauses to record memos for any insights that occur during the analysis.
\subsection{Memoing}


(data analysis) The flip-flop technique (Thornberg) can be used to look at data and codes to ask, what would happen if this didn`t happen. It enables the researcher to challenge preconceived ideas. ``What would happen if the opposite occurred" and examines the situation ``inside-out" (Strauss and Corbin 1998)

\subsection{Memo Sorting}
\subsection{Theoretic Sampling}
Theoretical sampling is collecting additional data to develop full and robust categories, identify the relationships between categories, and flush out the main category`s properties. Data and codes may yield unanswered questions and categories may not be definitive and may suggest new avenues of exploration. Additional data collection, coding, and analysis refines the emergent theory which produces a new vantage point for further exploration and refinement.

For example, data and codes may yield unanswered questions and categories may not be definitive and may suggest new avenues of exploration. Additional data collection, coding, and analysis refines the emergent theory which produces a new vantage point for further exploration and refinement.

Activities include adding new participants, observing in different settings, re-interviewing participants with follow-up questions or ask about different kinds of experiences.

This process continues until the category structure stabilizes and nothing new can be added to the theory, i.e. theoretical saturation. 

Memo-writing spurs theoretical sampling. Theoretical sampling is strategic, specific, and systematic.

\strikeout{Theoretical Sampling is not}
\begin{itemize}
\item Sampling to address initial research questions 
\item Sampling to reflect population distributions
\item Sampling to find negative cases
\item Sampling until no new data emerge.
\end{itemize}

\strikeout{While Quantitative Researchers aim to have a broad sampling to statistical inferences that describe target populations, qualitative researchers aim to fit their theory to their data. Quantitative researchers aim to test preconceived hypotheses, qualitative researchers aim to generate emergent theories that then become the foundation for new endeavors for quantitative researchers.}

In review, theoretical sampling is collecting additional data to develop full and robust categories, identify the relationships between categories, and flush out the core category`s properties.

\subsection{Evaluation}
For researchers who do want to generate theory, there are four concerns to attend to theoretical plausibility, direction, centrality, and adequacy. These concerns are the prime concern for the researcher.

Stohl et all define this as….

Since interviewing allows the researcher to direct and control the data generation, this brings up these theoretical concerns. 

Theoretical plausibility supersedes theoretical accuracy. Charmaz reminds that definitions of accuracy are social constructs. Constructivists express less concern about the accuracy of the participants point of view than other qualitative researchers.

The amount of data collected in a grounded theory study typically offsets any adverse effects of ``misleading accounts" and thus decreases the probability that the research`s work would have spurious results. ``GT aims to make patterns visible and understandable." Thus a grounded theorist should strive to have wide and deep coverage of their categories. 
If a participant does offer exaggerated or inaccurate accounts, and the researcher detects this situation, then this can be a research opportunity into understanding how the candidate is creating fictional representations of their situation. Charmaz recounts seasoned citizens retaining identity patterns from earlier in their life; for example, one person described how she daily attended her garden even though she had not done so for years.

After the initial interviews and the initial coding, a theoretical direction emerges from the data. Certain ideas and codes routinely emerge from the data. The data suggests paths that need exploring.

Once the the researcher has explored these paths, theoretical centrality emerges from the coding and analysis. Certain codes become core to the research. Less fruitful paths and codes are dropped as the researcher focuses interviews on the central theme or categories. 

During later interviews, the researchers introduces questions to assess the theoretical adequacy of the emergent categories for the purpose of robust theoretical cateogries. The theoretical adequacy is central to theoretical sampling and saturation. 

Objectivist grounded theory argues for note taking during the interview and argues for recording the essentials without getting lost in the details. (Glasses 1978, 2001) (I find this approach appealing for specific focused ideas such as determining the purpose of the IPM as I can quickly ask many people for their input until I saturate my responses.) A constructivist finds value in the details. Recording the interview enables the interviewer to enter into the narrative and later . (I have found that analyzing the data later helps me identify some of my own preconceived ideas. In listening to the audio and reading the transcriptions, I occasionally wince as I realize that I ignored something the participant said or interpreted what they said in my own frame of reference.This enables me to become a better interview as I now have a feedback loop.) 

Constructivist grounded theory specializes in breaking down phenomena and reflecting on dialog with `what` and `how` questions.

Charmaz contends that Glazer might say that questions like ``As you look back on your illness, which events stand out in your mind?" as forcing the data whereas she believes that it opens up common place aspects of life.

``the conversation includes more than words alone"



\subsection{How much data to collect?}

What makes up great data?
Two key aspects of data for depicting empirical events are suitability and sufficiency. (What does this mean??) Data should give a full picture. Plan to gather sufficient data. It is ok if everything seems trivial or everything seems significant.

For Glasser 1998 and Stern 1994a, small data samples and limited data is not an issue since Grounded Theory purpose is to create conceptual categories. (Site SF paper on static analysis tools.) Charmaz argues that limited data can lead to weak analysis. 


Charmaz points out asking how many interviews is enough is asking the wrong question. She`d prefer to have enough interviews that makes the research deep and with sufficient vigor. Studies using mixed methods may require less interviews.




\subsection{Dealing with biases}
\sout{combine with forcing data}
Each part of the research process (transcripts, annotations, analysis) can be checked for accuracy by other research team members. 


\subsection{Constructivism}
The Constructivist approach to Grounded Theory \quotes{emphasizes understanding and acknowledges that data, interpretations, and resulting theory depend on the researcher`s view} \cite{StolGTinSE}. 

The constructivist acknowledges that the \quotes{humanness} of the researcher may affect the research. Constructivists attempt to identify their assumptions and not accidental reproduce the assumptions in the research results. Relying and listening to the participants helps counteract researcher bias.

Because Charmaz has a constructivist perspective, she acknowledges that ``neither observer nor observed come to a scene untouched by the world." The research method affects the kind of data observed, the researcher`s background affects the observations that the researcher can see. The onus is on the research to bring scrutiny and reflective practice to understand how our own point of view may bias observations. Chamaz points out that ``\textit{people} construct data" through field notes, interviews, It is tempting to treat a report or document as fact, however people collected and formed them.

The contrasting positions of constructivist and positivists made me wonder about my own personal stance and how it might affect the research process. 
I believe that there are absolute facts to be obtained in mathematics and the sciences, whereas understanding people is a messy, organic process. For software engineering research, constructivist grounded theory seems aptly suited.

\textbf{Interviews}
Interviews are situated within a particular context, changes in location, time, and setting would result in different data. 

Sometimes asking a follow-up question such as \quotes{could you tell me more or what did you mean by \underline{\hspace{0.8cm}}} will result in the interviewee provided additional, valuable information.

Charmaz argues that interviews are a construction between the interviewee and the interviewer. ``The result is a construction - or reconstruction - of a reality. Through constructing their respective performances, interviewers and interview participants present themselves to each other. However silent, both the interviewer`s and participant`s performances make and negotiate identity claims." 



\subsection{Tool Support}
I initially tried various qualitative research tools, but found that instead of making me more intimate with the data, they provided a layer of separation. I found digital forms of simple techniques to more effective. I relied on Microsoft word for transcripts, and Google spreadsheets for constant comparison, printing to index cards for key or massive constant comparison.

I had a license for NVivo and I tried to make the tool work for me. Adding codes was not a streamlined operation. There is no keyboard shortcut to add a code and there is no auto-complete with existing codes in the system. As the researcher, I had remember all the codes you have already added to the system. The cognitive overhead of remember of o was adding a new code or re-using an existing code was too much for me. Perhaps the tool would be better for focused coding with a fixed set of codes.Furthermore, the interface to have my advisor review my coding was extremely difficult to use. The existing coding strip implementation need considerable overhauling.

I also tried Atlas.ti which has keyboard shortcuts and auto-complete suggestions on existing codes creating a better initial coding workflow. I did not like that I could not edit the original source material. I was coding a transcription and noticed and error. There is no way to easily change the mistake. 

I used Microsoft Word for transcripts and initial coding. I used the three column format. In the third column, I transcribed the interview, breaking sections into new rows. I compulsively recorded the time stamps for each segment, although  I maybe once or twice used that information to re-listen to a portion of the interview.  I did initial coding and focused coding in the the second column while registering to the audio file. I used the first column for a unique id reference for that rOw for the purpose of cross-referencing during constant comparison. 

I then exported the word document into a Comma separated format with the id and coding columns. These were then imported into a spreadsheet for constant comparison and analysis. Sometimes this was easy to do in the spreadsheet. On a few occasions, I printed index cards so that I could physically sort the cards around me. I use mail merge to print specific rows into a Microsoft word template for Avery shipping labels. Once these were printed, I would apply the sticker to color coordinated index cards. After sorting and clustering, I would update the spreadsheet to match the physical arrangement of cards.

At the outset of the research study, I was concerned about easily moving through my data. Using a spreadsheet to track constant comparison and flipping back and forth to various word documents, I was worried that it would be inefficient (whereas a tool would allow me just to click on a link.) however, in practice this proved less problematic.

\subsection{Similarities with Agile Software Development and user reseach}
As an iterative research approach, Grounded Theory shares similarities to agile software development. Both embrace the need for change and start without knowing the final destination. We plan only the next step and adjust our course as we acquire new knowledge and better understanding of our environment. Just like agile, we need to grow comfortable with ambiguity. 

The approach is similar to Pivotal`s Discovery and Framing process for user research. Interaction designers interview prospective users of the system, attempting to create and validate a persona(s). During each interview notes are written on post it notes. After a few interviews, the team does a dump and sort, creating buckets by placing post it notes into them, and doing constant comparison of each bucket. Key insights are noted. This is a pragmatic version of grounded theory that allows preconceived ideas and assumptions to be explicitly tested by explicit questions (the data is forced) and the sampling size tends to be less. Since the process is attempting to do user validation, less interviews seem necessary than needed to create an emergent theory.



\subsection{Summary}
Starting a grounded theory research study, the core question is `What is happening here?` (Glaser, 1978)

Grounded Theory allows new aspects of the research to be added while gathering data and can even happen late in the analysis. The data collection process is guided by the research and altered as knowledge is accumulated. It is common for early research to illuminate new angles.``Grounded theory can give you flexible guidelines rather than rigid prescriptions." The theory provides a starting place for inquiry, not a specific goal known at the beginning of the research. The researcher`s guiding interests can be used ``as points of  ``rture to form interview questions, to look at data, to listen to interviewees, and to think analytically about the data."

After interviewing, data analysis begins with line-by-line coding as recommended by Charmaz \cite{Charmaz}. Coding line-by-line helps the researcher identify nuanced interactions in the data and avoid jumping to conclusions. The data then advanced from these initial codes to focused codes, focused codes to core categories, and core categories to an emergent theory. 











For other chapters:
The initial interviews were open-ended explorations starting with the question, \quotes{Please draw on this sheet of paper your view of Pivotal`s software development process.} The interviewer specifically did not force initial topics and merely followed the path of the interviewee. While exploring new emergent core categories, whenever possible, we initiated subsequent interviews with open-ended questions. The first author transcribed each interview with timecode stamps for each segment. These interviews were spread across the duration of the research study. 
