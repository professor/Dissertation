\documentclass[oneside,letterpaper]{memoir}

\usepackage{citthesis}

\usepackage[T1]{fontenc}
% Load lmodern for bold \ttfamily
\usepackage[]{lmodern}
%\usepackage[lighttt]{lmodern}
%\usepackage[bitstream-charter]{mathdesign}
%\usepackage[urw-garamond]{mathdesign}
\usepackage[sc]{mathpazo}
%\usepackage{fourier}
%\usepackage{lmodern}

\usepackage{stmaryrd}
\usepackage{graphicx}

\usepackage[colorlinks,linkcolor=blue,filecolor=blue,citecolor=blue,urlcolor=blue,backref=page]{hyperref}

%\usepackage{caption}
%\let\subcaption\undefined
%\let\subfloat\undefined
\newsubfloat{figure}

%Todd added for fractions
\usepackage{xfrac}
%Todd added this for strikeout \sout{}
\usepackage[normalem]{ulem}
%Todd added this for the reference to essence.sv.cmu.edu
\usepackage{hyperref}
%Todd added this to fix tables with an H
\usepackage{float}
%for no orphan lines
\usepackage[all]{nowidow}
 %for lableling a table as a figure
\usepackage{caption}

%Todd's commands
\newcommand{\strikeout}[1]{\sout{#1}}
\newcommand{\quotes}[1]{``#1''}
\newcommand{\participantQuote}[1]{\textit{``#1''}}
\newcommand{\singleQuote}[1]{`#1'}
\newcommand{\emphasis}[1]{\emph{#1}}
\newcommand{\ignore}[1]{}

\newcommand{\oneColumnWidth}{3.4in}
\newcommand{\twoColumnWidth}{7.1in}



\setauthor{Todd Sedano}
\settitle{Empirical Study of Iterative Software Development in Academia and Industry: Effectiveness, Optimization, and Extension of the Essence Kernel}
\doctors
\setchair{Dr.\ David Brumley}
\setdept{Electrical and Computer Engineering}
\setdegrees{B.S., Computer Science, Millersville University\\
M.S., Information Security, Carnegie Mellon University}
\setdefdate{May 2014}
\setgraddate{May 2014}
\setcopyyear{2014}

\begin{document}

\frontmatter

\thetitlepage
\copyrightpage

\section*{Acknowledgements}
Thanks!

\newpage
\section*{Abstract}
Software development continues to be a complex endeavor involving many disciplines and skill sets. Practitioners and researchers experiment, research, and adopt practices to simplify, understand, and create effective processes. 

Given the plethora of practices and methods, the Software Engineering Method and Theory (SEMAT) community created the Essence kernel as a unifying framework for describing and analyzing software engineering endeavors. 

My research goal is to evaluate the Essence kernel for practical use on academic and industrial software development projects, identify issues, and research solutions grounded in empirical evidence. 

At Carnegie Mellon University in Silicon Valley, I conducted a field study with masters of science in software engineering students as they completed team-based capstone projects using the Essence kernel. During weekly Essence Reflection meetings, the Essence kernel checklists helped students identify relevant goals to achieve, which enabled the team to steer the project to higher states. The student teams found value during project inception. However, teams found less value during the construction phase of iterative projects, as the Essence kernel offered few new goals hence loosing its ability to help the team steer the project. The original Essence kernel is method agnostic and does not directly support iterative development. Since most of agile software development occurs via iterative software development, adapting the Essence kernel to have goals for an iterative construction phase would increase its value to software development teams.

Following these results in academia, my objective is to continue my research in industry, with a focus on the practices at Pivotal. One of my next goals is to generate a process model that accurately describes iterative software projects at Pivotal. My plan is to conduct participant-observation of several software development projects at Pivotal. I will interview many software engineers and product managers to collect additional data. I'll iterate my research by incorporating this feedback. My expected result is a process model grounded in empirical data that supports iterative software development. 

\newpage
\tableofcontents
\listoffigures
\listoftables

\mainmatter


%%
%% Start line numbering here if you want
%%
%%\linenumbers

%note that import will do a clearfix
% \input{essence-reflection-meetings}
% \input{essence-steering}
% \input{essence-green-lighting}

% % \begin{table}[h]
% \centering
% \renewcommand{\arraystretch}{1.5}
% \caption{Concise comparison of Grounded Theory Approaches}
% \label{ConciseGroundedTheoryComparison}
% \begin{tabular}{|p{1.3in}|p{1.5in}|p{1.5in}|p{1.5in}|}
% \hline
%                                     & Classic Grounded Theory                                                                              & Straussian Grounded Theory                                  & Constructivist Grounded Theory                                                                      \\ \hline
% The influence of research questions & emerges from the research                                                                            & may be defined upfront                                      & may be defined upfront and evolves through study                                                    \\ \hline
% The role of existing literature     & delays use of literature in the process                                                              & use when needed                                             & use when needed                                                                                     \\ \hline
% Analytic coding                     & Theoretical Coding                                                                                   & Axial Coding                                                & Theoretical Coding                                                                                  \\ \hline
% Analytic questions                  & \quotes{what is this data a study of?}                                                             & hypothesizing as to causes of the data                      & \quotes{what is this data a study of?}                                                            \\ \hline
% Philosophical differences           & objectivism                                                                                          & pragmatism                                                  & social constructionism                                                                              \\ \hline
% Evaluation criteria                 & fits the data, works in explaining main concern, relevance to participants, modifiable with new data & Seven criteria for process and eight criteria for grounding & credibility (enough data), originality, resonance with participants, usefulness with intepretations \\ \hline
% \end{tabular}
% \end{table}

\input{team-code-ownership}
% \input{sustainable-software-development}
% \input{software-engineering-waste-do-not-edit}

\appendix
\chapter{Appendix Chapter}
\section{Appendix Section}
Test in main

\chapter{Interview Transcriptions}
% \section{2015-05-29 Product Manager Interview}

\textbf{Todd:} Thank you. 00:01

\textbf{Interviewee:} Yes. You're welcome. 00:02

\textbf{Todd:} I was hoping you could describe your typical day. 00:04

\textbf{Interviewee:} My typical day, okay. We start every morning with stand up. It takes about five minutes. And that's where we go over what we did yesterday or what we're going to do today, any blockers or anything like that. I think the format differs from project to project but usually that's how I like to do it. 00:24

\textbf{Interviewee:} Then, I'll go through emails just to have some alone time and just go through my emails and things like that. And if I'm working with a client PM, we will then start going through the backlog. We might pair on writing stories. We might prioritize the backlog. 00:44

\textbf{Interviewee:} Usually, during the beginning of the week, we'll have an iteration planning meeting. It takes an hour and that's where we'll go through and go over some of the stories we've prioritized in the backlog and the dev will then point them or estimate them. 01:03

\textbf{Todd:} Thank you. Anything else? 01:08

\textbf{Interviewee:} There's more of the same stuff. There's lunch and then I might have a meeting or two sometimes just some internal sort of product meeting. All the PMs might meet once a week, once every other week or something like that. 01:22

\textbf{Todd:} We all say we do iterative development. For your perspective, what makes us iterative? 01:29

\textbf{Interviewee:} What makes us iterative is that we don't fully flush out a feature or anything like that upfront. And so we just take a first pass at it and get the basic functionality of it down and then layer on  top of that. 01:48

\textbf{Interviewee:} We might add some improvements or we may add some styling or we might add different things like that but we don't do that all upfront. Now, we can just get something workable  done. 02:02

\textbf{Interviewee:} Ideally, in front of users usually by that point, we're not, in the beginning, we're not putting anything in front of production or anything like that but that's the idea. We potentially could, which I think makes it iterative. 02:16

\textbf{Todd:} Thank you. This is my first drawing exercise for you. So, pretty open-ended question, could you describe a project work flow by drawing it on that sheet of paper? There's no wrong answers. 02:32

\begin{figure}[h]
\centering
\includegraphics[width=6.5in]{interviews/drawings/2015_05_29.png}
\caption{\quotes{2015-05-29's drawing of a project work flow}}
\label{2015_05_29}
\end{figure}

\textbf{Interviewee:} Does it matter if there's a DNF first or...? 02:39

\textbf{Todd:} However you want it. Typical, ideal, however you want to draw it. 02:44

\textbf{Interviewee:} We actually show this to clients. We have this drawing here. 02:54

\textbf{Interviewee:} Most of the projects that I've been on start with the DNF. We'll call this DNF. 03:07

\textbf{Interviewee:} It's assuming the project has design in development going. This kind of, like, inception. 03:27

\textbf{Interviewee:} And this is dev inception. 03:32

\textbf{Interviewee:} This is your discovery, and framing. 03:50

\textbf{Interviewee:} Here in this area, we're sort of doing more exploratory research and sort of going wide trying to understand the users. That's something we do at the beginning of every project especially if the client doesn't know who their user is or they have ideas of who they are but it's not validated. 04:10

\textbf{Todd:} 	That's the discovery phase. 04:11

\textbf{Interviewee:} Yes. So, this is like exploratory user research. That's like interviews and maybe on site, shadowing people and things like that. And once we have a good idea of who they are, we start wire framing, maybe like the main flow and doing some wire frame tests like user research, that kind of thing and that's kind of within that, I think, we're kind of iterating as well. 04:45

\textbf{Interviewee:} I guess you can either start with one and iterate on that or you can start with many and narrow down. We've done both. So, when you start with many, you just- So, this is like your different As, research, then research to get to C. That makes sense? 05:10

\textbf{Todd:} It's like a portfolio of ideas? 05:14

\textbf{Interviewee:} Yes. I did it on my first project and it worked out really well and then from there you're kind of iterating on what you've come up with. 05:25

\textbf{Interviewee:} We came up with a couple, different nuance ways of doing something because they all seem good and they're like, two or three different features and we sort of combined them in different ways and then we took our insights from that and narrowed it down two versions, narrowed it down to one version. 05:46

\textbf{Interviewee:} I've also done it where you just start with A and you just iterate. I think both worked but this is fun to do. By the time we get to dev inception, we have prioritized epics. You have a develop persona. And you have wire frames. That's kind of the ideal for this point. At that point, development can start on say feature one and in the meantime, we'll start researching feature two. Then, we'll develop it. Then, feature three. So, it kind of goes like that. 06:37

\textbf{Todd:} Nice. 06:39

\textbf{Interviewee:} Maybe you have feature one here and then we'll go here. While they're doing that, we'll start on the next thing. So, on the next thing it flows down. I hate saying waterfall because I know it's the wrong kind. It's like a different kind of waterfall but it iterates in that way. So, we're like, \quotes{Oh, it's going in the cycle.} At any time, we're going to bring in users to test whatever we want. 07:09

\textbf{Todd:} You're drawing people. 07:12

\textbf{Interviewee:} Yes. These are users. At any given point, we can even do exploratory research. We can do wire frame or visual research or we can show them an actual prototype. 	It's kind of nice because you can, whenever you need people, whenever you're stuck on something, you don't have to wait for the right time to bring people in. 	It's just you can test anything at any given time.  07:34

\textbf{Todd:} Thank you. Have you seen a project that didn't quite fit this, what you've drawn here, this model? Or it was really different? 07:46

\textbf{Interviewee:} Yes. I was on a project. I was only on it for two weeks. It was just really quick enablement but there's basically no DNF. They came in with full visuals, full design. The problem is not a lot of it was validated. A lot of it was like, \quotes{Oh, the stakeholders thing. This looks nice.} 08:02

\textbf{Interviewee:} That was kind of different. As a PM, it was weird to write stories for full visual designs because I wasn't sure where to draw the line with the styling and things like that. We kind of had to feel it out. I think we stopped at colors and 	fonts. 	08:19

\textbf{Interviewee:} We did colors, fonts and a little bit of spacing for some other things but no animations, nothing like that. And that's also hard to convince the client, that maybe this isn't the right solution or things like that. I'm not sure how they got 	through without validation but it happens sometimes. 08:43

\textbf{Todd:} Do you think because it had such detailed mocks, they were married to their ideas? 08:49

\textbf{Interviewee:} I think so. It wasn't too hard, they're actually pretty open minded about stuff because they trust at our process. A lot of clients come in and like, \quotes{We trust you.} 08:58

\textbf{Interviewee:} For the most part, all the projects I've been on have been very flexible from that standpoint but I definitely felt bad for making the designer go through stuff. I don't think he minded but it's just a little bit harder to push back on things when they're not black and white and flexible. 09:21

\textbf{Todd:} Can you think of a different project that maybe didn't fit this mold in a different way? 09:27

\textbf{Interviewee:} Not out of the ones I've been on. They've mostly followed this. Like my first project followed this to a T. It was great. The second one just didn't have any discovery or framing. Right now, we kind of follow this as well. 09:51

\textbf{Todd:} I think given our sales process, we set this up for us. 09:56

\textbf{Interviewee:} You mean like non labs projects. 10:02

\textbf{Todd:} Just some labs. 10:02

\textbf{Interviewee:} 	Oh, just some labs. 10:03

\textbf{Todd:} Or in Pivotal. 10:06

\textbf{Interviewee:} Yeah. I haven't heard of any that- I think there are some enterprise projects that this doesn't quite work because there's an added QA layer. I haven't worked with a QA team but it just makes it harder to follow this because there's a QA layer that you have to go through in addition to acceptance. 10:28

\textbf{Todd:} In my current project, there's an approval process on designs and they want to see everything. You can't really do the features incrementally at least on design level. 10:40

\textbf{Interviewee:} That sucks. We have some design reviews. On most of the projects I've been on, we had weekly design reviews so nothing was a huge surprise to people. I guess I've just gotten lucky with the clients. They've been mostly receptive to our process. I hear horror stories about it but I haven't personally experienced any huge ones just yet. 11:09

\textbf{Todd:} As a PM, it feels like there's a lot that you juggle and manage. Is there one or two things that you remind yourself each day? What's the most important thing that you try to get right on a project? 11:26

\textbf{Interviewee:} If I'm working with a client, I always have to think about what I'm going to do with them. I think that's the biggest worry because I think any of the stuff I can do myself really quickly and I'll just remember things as they come up but I have to remember that I have to enable the client. 11:42

\textbf{Interviewee:} I have to be cognizant of what are we going to do today? What does the week look like? Not just the PM but I think PM in design especially during this period. It really helps to plan out the week just roughly and figure out okay, when are we doing research? What are some activities we should do? 12:01

\textbf{Interviewee:} There's some things you have to think about with the client from their perspective. Maybe they're not open to something. How do I convince them that this is the right way and try to put myself in their shoes and have empathy for them so I know how I can help turn them around to something that can work. Or help them come up with the idea. And I'm just like, \quotes{Oh you came up with this idea.} That kind of thing. 12:29

\textbf{Todd:} You mean you had the idea and you helped them realize the idea as well? Is that what you were saying? 12:35

\textbf{Interviewee:} Yes. We wanted them to go a certain way but they were like, \quotes{No we have to go this way.} How can we make them think that they came up with going this way. 12:43

\textbf{Todd:} It's like inception. You plant seeds. 12:45

\textbf{Interviewee:} Yes because they only listen to you so much. You tell people, \quotes{Trust me.} But they'll only trust you so much and then when they actually experience it, sometimes you just have to let people make a mistake and they feel that pain and they're like, \quotes{Oh okay. I will never do that again.} maybe not do that again maybe try this way. \quotes{Let's see what happens with that.} It's shows a different way of thinking. 13:12

\textbf{Todd:} What's it, for instance, on letting the client make a mistake and letting them feel that? 13:17

\textbf{Interviewee:} One time we had some clients who were very, very tight to us that are features and they call them core capabilities and very concrete and final and we kept trying to say, \quotes{Okay. These are assumptions.} They just didn't understand the word assumption. They're like, \quotes{No. We need to build these things.} 13:34

\textbf{Interviewee:} So, we did a couple of rounds of research because they weren't down for research. They never done that before and once they heard certain things, I think even though they heard certain things, their brain was still telling them they want these core capabilities. So, it's like they were, what is that called, like a confirmation bias or something like that or they're just like, \quotes{Oh yeah. That's what it means.} 13:53

\textbf{Interviewee:} And so, we did a synthesis and mapped out everything out and took their core capabilities and our main insights and tried to match them up. I think from there we could see that a lot of the insights didn't map to the core capabilities but it wasn't aggressive. We were all doing it together. 14:12

\textbf{Interviewee:} They were the ones making the mappings and saying like, \quotes{Oh. This moves down in priority. We didn't expect this to come up. Let's move on this first.} It took more time but it was nice that they came up with those conclusions on their own in a way or able to see that rather than us just telling them don't worry about this. Focus on the research. 14:37

\textbf{Interviewee:} You have to remember these clients come up with these things for months or years so they're very tied to them and to just rip them away from that is insensitive. I can empathize with them. You just have to have slowly validate or invalidate the stuff that they've been so closely tied to. 14:58

\textbf{Todd:} You described the client relationship and the parties you have with that. When you think about the development team, I'm biased, how to set them up for success, what's sort of things do you try and enable or set up for the team? 15:13

\textbf{Interviewee:} For the developers specifically? 15:15

\textbf{Todd:} PM. They're trying to make everything hum than what you imagine. 15:19

\textbf{Interviewee:} I keep communication really open. Before I start a project, if I can I like to talk to whoever is the anchor or the designer in the team and just get a feel for how they work with the PM because everybody's a bit different. 15:36

\textbf{Interviewee:} Big things that they would expect from a PM that I might miss or something like that and just try to get a feel for how they work with other people and obviously retro is great for a lot of that stuff. 15:50

\textbf{Interviewee:} Sometimes, we've had retros where we just talk about- because I think in front of the client you have to have a united front. And for the most part, in my experience, it's been great but if there are any disagreements or any problems, you don't want to bring up in front of the client it's really nice to have a pivotal retro. 16:09

\textbf{Todd:} How frequently might you have one of those? 16:12

\textbf{Interviewee:} Depending on how bad the project is. Once a week, maybe. Maybe once every other week. I don't know. 16:19

\textbf{Todd:} On top of the normal retro? 16:21

\textbf{Interviewee:} On top of. We have the normal retro usually and if we were at a point where we're feeling disconnected, we might have one. So, it kind of depends. I've never been in a situation where I needed one weekly. Although I've heard of those, really stressful ones. 16:39

\textbf{Todd:} Think of one of the worst projects you've been on, what's one thing you did to try and right it and get it back on track? 16:47

\textbf{Interviewee:} I haven't had any that were horrible. I think there was some bad times in my first project because the client wasn't committing to certain things. Once they went back to their own office. 17:10

\textbf{Interviewee:} We're working remotely and they were just being pulled into meetings. I think I was a little harsh. I don't know. I was just like, \quotes{Hey, the backlogs' drying, designing guys need to do stuff.} Because the devs were running out of things to do. They're just like, \quotes{What do we do?} 17:27

\textbf{Interviewee:} I tried to find other things that didn't require design to be prioritized. For that, I needed some help from the anchor because I don't have a formal technical background. 17:39

\textbf{Interviewee:} A lot of times I'll just ask. I'm like, \quotes{Hey, teach me this stuff.} Sometimes I'll have the devs say in stand-up that the backlogs' dry because I find the people listen to devs more than PMs or they take it more seriously. So, I'm just like, \quotes{Okay. You just say it.} I can say it 10 times but as soon as the devs say it, they're like, \quotes{Oh my god. [Interviewee] We're freaking out.} And I was just like, \quotes{Yes. I was freaking out two weeks ago.} 18:09

\textbf{Interviewee:} Sometimes you just have to be a little harsh. Just be like, \quotes{I know you have a meeting but we really need to move on this right now.} 18:21

\textbf{Interviewee:} I also paired with devs and designers which I think helped bring the team a little bit closer. The client PM was there half the time but I encourage him to pair sometimes. I really like cross functional pairing. 18:35

\textbf{Todd:} When you pair with dev, do you work on a story with the dev?18:39

\textbf{Interviewee:} Yes. I do whatever they're doing and I'm not necessarily coding from scratch but I know enough to recognize where things go. We'll type this in here or I might watch them do something then I'll do it. 18:52

\textbf{Interviewee:} A lot of it is discussion and asking questions. Like on this particular time, I had been on a project longer than the dev and so she wasn't sure where everything lived and I was like, \quotes{I think that data is over here. Can we do control+F for this keyboard} We're trying to figure out why something didn't work and I had a little more context so I was able to talk about it and she's like, \quotes{Oh. I know where that is. Let's try doing that type of thing.} It was a lot of back and forth. 19:22

\textbf{Todd:} Nice. I have not experienced that. 19:25

\textbf{Interviewee:} Yes. It's really fun. I love pairing with devs. It's also really intimidating. If you're at the helm of a spaceship or something and you're just like, \quotes{Oh my god.} but it's pretty fun. Everybody I've paired with has been really nice and patient and they like teaching. I was like, \quotes{Cool. Teach me.} 19:41

\textbf{Todd:} Pairing with designers makes sense to me, I think. 19:48

\textbf{Interviewee:} Yes. Again, it's a lot of discussion. Maybe we should put this here instead of there. I didn't know how to use Illustrator before but know I can poke around and use it. So, sometimes I get to do some stuff too. 20:01

\textbf{Todd:} That's fantastic. 20:03

\textbf{Interviewee:} Yes, it's really fun. 20:04

\textbf{Todd:} So, as a dev, sometimes I think about non functional requirement or they're called quality attributes. How do you handle those? As a PM, how does that work for you? Do you think about them? 20:19

\textbf{Interviewee:} You mean like visual design, thing like that? 20:23

\textbf{Todd:} Things like maybe performance or security or liability of the system. We call them ilities. There's like 20 of them that might show up in a project. 20:31

\textbf{Interviewee:} Yes. I think about them a little bit, honestly, in my experience here, at least. It's usually the devs or the client PM who is bringing it up. I should think about it more probably but when I think about MVP, it's like user mode but I think those get prioritized. 20:49

\textbf{Interviewee:} We just feel it out and honestly it's more of the client PM's decision and because they know the product better. If they're really, really lost, we'll try to make the most informed decision with them. 21:02

\textbf{Interviewee:} If we're about to launch or we want to do an initial release and obviously some of that stuff has to go forward. If some things, like really unknown or there's high risk, sometimes we try to write a story and just get that one story out there because that drives out a lot of the risk and drives out a lot of things that we might not know. 21:21

\textbf{Todd:} Trying to reduce risk makes sense to me. In my experience at Pivotal, I've not seen any stories about performance or things like that. That makes sense at some level because stories are about features. They're not about attributes of the system and I haven't seen how anyone here handles that. I'm curious... 21:44

\textbf{Interviewee:} I try to write all stories from a user perspective so I might say something like, \quotes{I want the spatial load faster because it's taking a really long time right now.} That kind of thing and so I know some places have requirements of loads in two seconds and things like that. 22:01

\textbf{Interviewee:} I haven't quite gotten down to that level but I'll still write it from a user perspective and if I have an idea of what might be slowing something down for example, I'll put it in there. \quotes{Oh, we have a page full of SVGs maybe we can convert them on to PNGs or something like that.} I try not to be too prescriptive because that's just not my place. 22:24

\textbf{Todd:} Good. 22:27

\textbf{Interviewee:} I try to handle them from the user perspective as much as possible. 22:30

\textbf{Todd:} I have created a little model for stories. This is now called described or something. Defined. 22:44

\textbf{Todd:} From your perspective, this is a flow of stories, give me feedback for me on those. 22:53

\textbf{Interviewee:} Named is just identifying the story. Once you flush them out and written them and they've been through pre-IPM and they're estimated, they're started, blocked. I usually don't go back and estimate. So, it's like this. 23:12

\textbf{Todd:} It could happen. 23:13

\textbf{Interviewee:} I usually don't do this. I guess it depends on what's happening here. I haven't re-estimated a story unless we pointed it five weeks past and then it comes up in the backlog, we might revisit it and say, \quotes{Okay. Do we still think it re-points?.} 23:35

\textbf{Todd:} Well, we try not to re-point. I agree with that. 23:37

\textbf{Interviewee:} I haven't done that. Started, finished, delivered, accepted. 23:42

\textbf{Interviewee:} Delivered. Checked in. This would go back to - Oh, you have that here. Delivered. So, when will it go from delivered to started?  23:56

\textbf{Todd:} The irony is this actually comes form tracker data, this graph. I went through a ton of project and watched all the transitions of every story and there was a story from delivered to started. 24:08

\textbf{Interviewee:} Maybe they delivered on accident? That sometimes happens. Or was it delivered on purpose? 24:14

\textbf{Todd:} I think what happens there is you deliver it and then you realize there's more work to be done. 24:19

\textbf{Interviewee:} Oh, so the dev will go back and... 24:20

\textbf{Todd:} Say, \quotes{Oops. We're not done.} 24:23

\textbf{Interviewee:} Okay. 24:24

\textbf{Todd:} Or someone will walk by and say, \quotes{That's not done. You didn't do X.} 24:28

\textbf{Interviewee:} Just like  before the PM hits something, they'll say, \quotes{Hey, you guys, forgot this.} 24:32

\textbf{Todd:} It's kind of like the rejection where the PM didn't reject it. 24:34

\textbf{Interviewee:} That makes sense. I've done that before. Finish, delivered, accepted. 24:40

\textbf{Todd:} That has happened, believe it or not. 24:47

\textbf{Interviewee:} Really? 24:48

\textbf{Todd:} I agree with you, the ideal is that would not happen. My point in showing this is to actually help you make sense of what I'm about to show you next. This is my view of stories. 24:56

\textbf{Interviewee:} I've been trying to create a checklist items for each state that the story could go in. I just wanted you to look at this list and see if you have any... 25:06

\textbf{Interviewee:} The idea is, for something to be named how do you know it's named or not and these would have to happen for each state. If this is not clear, ask questions and I'll try and clarify. 25:19

\textbf{Interviewee:} Okay. I think this is like an either/or. It could be prioritized or it could not be. Can we write down? 25:34

\textbf{Todd:} Sure. Whatever you want. 25:36

\textbf{Interviewee:} Defined. Feature is prioritized. I think this can also be either/or. 25:45

\textbf{Interviewee:} I usually don't prioritize until - I'll have a rough priority but I'll wait until they're all flushed out. Sometimes in pre-IPM even they're not fully prioritized by this time. Usually after they're estimated, usually they're prioritized but sometimes after they estimate, I'll rethink some things. 26:13

\textbf{Interviewee:} If they're not fully prioritized, that's okay. 26:16

\textbf{Interviewee:} Story describes clear acceptance criteria. That makes sense. That makes sense. 26:22

\textbf{Interviewee:} Story only describes user interaction with system. Yes. I think some people do this. I really like to do this. Oh, ideal PM. 26:33

\textbf{Todd:} That's your ideal. 26:34

\textbf{Interviewee:} Yes. 26:35

\textbf{Todd:} It doesn't have to happen. 26:35

\textbf{Interviewee:} Yes. Business value is clear. Yes. I'd say, relevant. Did you already write that? 26:59

\textbf{Todd:} No. I did not. 26:59

\textbf{Interviewee:} I try to attach design to every story. I try not to write a story before a design is done and that works out really well. And that's kind of the ideal. 27:09

\textbf{Interviewee:} Estimated. Feature is estimated. Nothing blocks developers from working on the story. Yes. I would even make this required. 27:24

\textbf{Todd:} I've estimated stories that were blocked that's why it's ideal. So you feel strongly about that. 27:34

\textbf{Interviewee:} If it's blocked, I don't. 27:36

\textbf{Todd:} Why do you estimate it? 27:37

\textbf{Interviewee:} It's not going to be accurate. I guess there are some stories where the devs are kind of like, \quotes{Well, I've never worked with that before.} If something's not defined or this is clearly blocked then I'll probably be like, \quotes{Okay. We'll just wait.} 28:00

\textbf{Todd:} Would you say it would be okay to estimate if you're trying to figure out when the end of the project is going to be even though things might be blocked? 28:06

\textbf{Interviewee:} Yes. I can see doing that. I've heard of that. I haven't don that on a project yet but I know they've been pointing  parties where people do small, medium, large type of thing or really rough pointing estimates. I think that's fine but for an IPM when you're going to work on something ideally, it's not blocked. Started. Yes. Blocked. It depends to you. Yes. I think that makes sense. I can't speak to these two. 28:37

\textbf{Todd:} You've paired on code so you might have more of an opinion on that than other PMs. 28:45

\textbf{Interviewee:} I agree with it. It makes sense. It was funny, I was pairing with somebody and we were working on an iOS project but it also involved JAVA and so we had, was it C\# or not C\#, what is it? 28:58

\textbf{Todd:} Objective C. 28:59

\textbf{Interviewee:} I guess she's more familiar with Objective C and not JAVA. She's kind of like, \quotes{Oh. I don't know Java that much.} She was like, \quotes{It all looks the same to me.} It looks equally confusing. 29:15

\textbf{Interviewee:} Units test pass, makes sense. Acceptance passes. What is the difference between a unit test and acceptance test? 29:28

\textbf{Todd:} Unit test are the very low level. They test one thing in isolation. Acceptance test, they're more PM level. Does the feature do what it's supposed to do? Does the story work? 29:42

\textbf{Interviewee:} So, the devs will do an acceptance test. 29:47

\textbf{Todd:} Sometimes they have them sometimes they don't it depends on the project so I'll probably put ideal or optional on there as well. 29:51

\textbf{Interviewee:} Is that something actually written or is that more of like the devs going through what they acceptance criteria says? 30:02

\textbf{Todd:} It depends on the project. Some of them we will write acceptance test. 30:08

\textbf{Interviewee:} Code reviewed prior to commit. Changes reviewed prior to commit. That makes sense. Delivered. Yes. PM updated the acceptance criteria because they were not clear. 30:28

\textbf{Todd:} You're going to fill out some of these. Go ahead keep going. 30:32

\textbf{Interviewee:} I do some of these things. Additional requirements that need to be implemented. The PM changed the requirements and the codes need rework. This shouldn't happen. This shouldn't happen. This shouldn't happen. 30:50

\textbf{Todd:} I completely agree with you. 30:52

\textbf{Interviewee:} If the PM changes their requirements, if it's small, sometimes, I'll say, \quotes{I meant this but I forgot to put it in.} If it's really tiny, they're like, \quotes{It's okay we'll just do it.} I'm like, \quotes{Okay. Thanks} 31:03

\textbf{Todd:} Why write another story. 31:03

\textbf{Interviewee:} Yes. And I'll just put a comment. I'll say, \quotes{It works but I forgot to add this.} So, devs rejecting so you can add this thing or something like that. But if it's something big, I'll be like, \quotes{Sorry, I forgot that.} And I'll just write another story. I'll be like, \quotes{Let me just write another story.} 31:23

\textbf{Interviewee:} Sometimes I'll write a bug and I don't have a clear criteria when it's a bug and when it's an extra feature but in the moment it makes sense. I just can't think what I used to decide it's a bug or not. 31:38

\textbf{Todd:} I know what I use. 31:43

\textbf{Interviewee:} One time, we were building something where you can adjust the quantity in a card and we were like, \quotes{It has to be an integer. it can't be zero.} And we were able to add zero and we were like, \quotes{We shouldn't be able to add zero.} That was a rejection but the fix they made, it had to be applied to this other scenario that wasn't covered in the story. But since they did it for one, \quotes{Okay. Let's just create a bug for it.} Because it was treated as a bug in the other scenario so it's kind of like a weird scenario. 32:21

\textbf{Todd:} For me, the black and white description is, if the story says it should do something and the code doesn't, that's a bug. If a story doesn't describe that condition, then it's a new feature. 32:32

\textbf{Interviewee:} Yes. 32:34

\textbf{Todd:} There's probably a gray area. 32:36

\textbf{Interviewee:} Yes. There are some gray areas where you decide should I reject it or should I write a bug type of thing. I think, definitely, in that first scenario, it's either reject a bug or usually that would just be a rejection because it says it should do this but it doesn't. 32:49

\textbf{Interviewee:} Anything additional should be another story but sometimes I do a bug instead of a rejection for some reason and I think it's for little things. PM updated the acceptance criteria because they're not clear. That sometimes happens. If it's small. So, I'll say, if it's small. And sometimes when I reject something if it's really nitpicky, I'll put nit. Do you guys do that? 33:18

\textbf{Todd:} What does that mean for you? 33:21

\textbf{Interviewee:} I don't know. A dev told me to do that one. I think it stands for nitpicky or nitpicking. I don't know. Sometimes, I'll be like the word is off or something. Technically it can go out but it should have this other word or the color is a little bit wrong and I'm just this works perfectly but I'll say it, nit. 33:44

\textbf{Todd:} So, in the comments of the story, you will put nit:fix this. 33:51

\textbf{Interviewee:} Yes. I feel it's a little nicer. I don't know. It's just like nothing's wrong, it's just the mock has this and it's a UI type of thing. A Dev told me to do that once. I was like, \quotes{Okay. I feel better about doing that because I feel bad rejecting stories that are technically work but just have a weird...} 34:10

\textbf{Todd:} I think I've seen one PM not accept it and put a comment on it,  which notifies people and then they would accept it once it's fixed. But rejection is probably the right flow. 34:22

\textbf{Interviewee:} Yes. If it's not rejected, you're not going to see it. As a dev, I would imagine. QA manager added new requirements. This should happen. I just think that's so silly but I guess some big companies have a QA process but they should not add new requirements. They should be bugs features and then... 34:55

\textbf{Interviewee:} I've had QA say, \quotes{Oh, this doesn't work.} or \quotes{This has this one case it won't take a credit card after 2049.} I was just like, \quotes{Okay. Let's write a bug} but I just put at the bottom backlog. We'll put them in the backlog but the PM will prioritize it as they prioritize everything else. That shouldn't lead to a rejection ever. 35:17

\textbf{Todd:} Given everything that we talked about in this conversation, is there anything else that you think would be relevant? Maybe questions I should ask? 35:26

\textbf{Interviewee:} What is your objective? What stuff do you want to know? 35:31

\textbf{Todd:} I should have a clear answer to that question. I think at this point, I'm really curious how what we do differs from what Kent Beck described 15 years ago. We've done a lot of really neat things and so I've been really interested in that. And I have a question I'd prefer to ask without leading it. Why do we call an IPM an iteration planning meeting? 36:10

\textbf{Interviewee:} Versus? 36:13

\textbf{Todd:} Why do we call it that? There's things that we do in the meeting and does the name describe what we do in the meeting? What's the purpose of an IPM? 36:27

\textbf{Interviewee:} It's so the team is aligned in what we're going to be doing for the next iteration or the next week. I guess, sometimes, the things that you point in IPM don't always line up to what you're going to be working on in a week. Sometimes, we'll end up pointing a lot of things. I think it's a good way. I think ideally, it is supposed to spell what the team is going to be working on for the next week or the next iteration. 36:54

\textbf{Interviewee:} So, you know you have immediate contact. You just pointed it. You know exactly what's happening. There's not a lot of time between for you to forget. Most of the time it happens or at least you'll start working on some of the things you pointed on that week but I think also it's a good rhythm setter for the team like you meet every week and you feel like you have this check in. You're always connected. You have frequent connections with dev and PM and design. Because I think PM and design can be off. We're usually ahead, design especially but PM, kind of split but we have fear what's going on right now and also be two weeks ahead and so I feel like this helps bridge the gap between what's going on in the design world and the dev world and having those check ins really helps 37:48

\textbf{Todd:} For me, it's like weekly estimation meeting. About six months into my career here, I realized that, for me, in my perspective there is no iteration. We do retros on Fridays because Mondays we don't remember what we did last week. We do weekly estimation meetings somewhere in the middle because we need to have enough work. The worst thing that could probably happen is I have six devs with nothing to do. So, we need to make sure that there's enough of a backlog for them. I was on a project where I had no idea that iteration started on Tuesday. And that was the clue for me. What? Are there iterations starting on Tuesday? I had no idea. 38:28

\textbf{Interviewee:} So, what do iterations mean for you then? 38:31

\textbf{Todd:} I think there is no spoon. I think it's a very kanban-like the work flows through us. The team isn't committing to a scrum that says, next week we're going to deliver these three features. 38:44

\textbf{Interviewee:} Yes. I can see what you mean by that. I feel that way, too, sometimes. 38:49

\textbf{Todd:} I think that's great. We show up and we do where the work is needed. 38:54

\textbf{Interviewee:} Yes. 38:56

\textbf{Todd:} So, halfway through a week something in the universe completely changes, we respond to it and start adapting to it. It's not like there's this ready plan. On Friday, we're going to launch this feature. We can't do that. In the original definition of XP, the team was committing to where they will be each week. 39:16

\textbf{Interviewee:} I didn't know that. 39:17

\textbf{Todd:} Yes. 39:18

\textbf{Interviewee:} I had no idea. I thought XP was totally. 39:21

\textbf{Todd:} And you couldn't change what the team would do that week? It was inflexible. 39:24

\textbf{Interviewee:} That's against everything that we do. I actually didn't know that and I read the XP book. 39:30

\textbf{Todd:} If the PM changed what they did during the first week, there was a contract violation between you and the team. 39:36

\textbf{Interviewee:} That's so crazy. I need to go re-read that book or something because I totally have been telling clients XP is flexible. 39:41

\textbf{Todd:} and..with XP..what we do here. That's really interesting. 39:45

\textbf{Interviewee:} Yes. That's kind of how I see it. I think, if anything, it gives a team a rhythm. I think retros help us sort of figure out what didn't go well during the week. What we need to change so it's not at some arbitrary point somebody might say, \quotes{This isn't working. Let's try to figure out how to change it} There's a forced check in so anything that you think you had a problem with this week, you know we'll come up at least every week and then you can do something to make the next week better. I think, work flow wise, maybe we're not doing things iteratively but team-cadence wise, we have iterations of like, Okay. Let's take an arbitrary piece of time. A day's too short, a month is too long. So, let's do a week. Just make sure there are teams running smoothly every week. I think it serves that purpose maybe more than the actual delivery and execution. It's more about the team spirit, I guess. 40:59

\textbf{Todd:} I think what I'm hearing you say is, your view of the word iteration might be different than mine and so you're happy to call in IPM because, for you, it's this weekly rhythm that's maybe artificial because of weekends. If there were no weekends, we might have a different rhythm or something 41:20

\textbf{Interviewee:} Yes. 41:21

\textbf{Todd:} It wouldn't be healthy 41:22

\textbf{Interviewee:} Yes. I think it's like a force check in because we were just to do things. We think about things in terms of time and so if we didn't have a time marker I think we would just get very confused and we would get very lost in what we're doing. 41:40

\textbf{Interviewee:} And even new devs are estimating stories, they still want to estimate in terms of time. We're always thinking in terms of time because that's how the world works in a way and so I think having that time marker helps us keep things in check versus if we just let them run on forever. 41:58

\textbf{Interviewee:} So, yes. I guess I think about iterations. I haven't thought about this deeply before but I guess me, for iterations, it's more like the team rhythm and team cadence versus the actual execution of things. 42:09

\textbf{Todd:} Yes. Okay. 42:11

\textbf{Interviewee:} I guess iteration in context of how we use it here. 42:14

\textbf{Todd:} That's very helpful for me. I'll have to deep think on it some more. 42:18

\textbf{Interviewee:} Building iteratively I think it means something else. Like what I was saying in the beginning, you just start with the first layer and the second layer. That's a little bit different than what I'm thinking in terms of team iteration if that makes sense. 42:31

\textbf{Todd:} It does. I don't have anything else. Is there anything else you wanted to share or anything you're not telling me? 42:40

\textbf{Interviewee:} I don't think so. I've worked in waterfall before so this is like a huge breath of fresh air. I think it's much healthier. I think the ideal of what this is supposed to be is good but it's not always applicable especially with enterprises, for example. We've definitely had to bend a little bit more to work with enterprises because it's just not the solution for everybody. You have to cater a little bit to different types of organizations. 43:19

\textbf{Todd:} I forgot about demographics. How long have you been at Pivotal? 43:21

\textbf{Interviewee:} I've been here almost a year. I started in August. 43:25

\textbf{Todd:} Just in terms of your career, when did you graduate from college? 43:31

\textbf{Interviewee:} 2010. 43:32

\textbf{Todd:} Great. That's it. 43:33

\textbf{Interviewee:} Okay. Cool. 43:34

\textbf{Todd:} Thank you. 43:34

\textbf{Interviewee:} Yes. Thank you. This was fun. 43:36


% \section{2015-08-12 Anchor Interview}

\begin{figure}[h]
\centering
\includegraphics[width=6.5in]{interviews/2015_08_12_anchor.png}
\caption{\quotes{2015-08-12 Anchor's drawing of software development process}}
\label{2015_08_12_anchor}
\end{figure}

\textbf{Todd:} 	So on a sheet of paper, curious could draw your perspective on our process for software development. 0:00:09

\textbf{Interviewee:}  	Our process for software development.  Where do you want to begin?  Or is that like…  0:00:14

\textbf{Todd:}   Really open.  0:00:15

\textbf{Interviewee:}  	So there's a customer, a potential customer.  And they come and there's some kind of vetting to, at least, get them in the door.  We think that they have interesting idea of some kind, and maybe we ask a couple of questions; so this might be the OD or sales.  Just some minimal vetting and then just enough for them to say: \quotes{you know what, I think we could possibly have a conversation about what it is that you want to have built.}  So then we do a scoping with them, and in that scoping, the input is their general idea.  Usually, it's like pretty vague; It could be from very vague to they totally know what they're talking about both on their problem space and with the solution to look like.  So then the output of that is a document that basically tells them this is what it likely is, and this is how much it's gonna cost, roughly speaking, and these are gonna be the roles and responsibilities, this is what you would be buying.  What just really actually, some amount of hours of some amount of people, different skill, that's what we're actually promising and we'll aim for this thing, but we'll always constantly trying to give you the best value thing as we go along.  So that's some kind of proposal, I don't know exactly what we call that, but it is all part of the scoping.  0:02:00

\textbf{Interviewee:}   And then, I'm gonna draw, like, maybe a number of stops here; by the time I see it, somebody signed, like, an SOW or similar that was possibly, like, but influenced by that scoping value but who knows what negotiations or whatever happened after that; and probably informed by that document, like, what we say we're aiming to deliver.  But this is, I think, this is more like a legal document or as the SOW is more like a legal document whereas the thing that came out of scoping is more of like a \quotes{come work with us} and some of the other stuff around that.  So then, we go into an all of these assumes it's ok.  I'm talking happy path.  Cool?  0:02:59

\textbf{Todd:}  	Yeah, I'm good.  0:03:00

\textbf{Interviewee:}  	So then we do an inception, and this is where, regardless of what we said over here, we get them to get really clear about who is the person that they're targeting; like, ok so it's about a product market fit, who's the market, what's the product gonna be.  So here's this person or persons or personas, and then for each of one those, like, what are the kinds of things that they need to get done, right?  So they've got things that need to happen, that ultimately will result in to some kind of outcomes, and it almost always like it's gonna convert to dollars in some way, shape, or form.  Sometimes it's a non-profit thing and that's slightly different, there's mixed motivation.  So then what we do is we say, \quotes{ok, in order to meet those tasks, what we need is features from the product} and so we're calling out, like, epics, if you will, feature sets, whatever, and then when we start breaking those down into individual stories and these are just sketches at this point.  We're not gonna get all into acceptance criteria right away, it's just like trying to enumerate coz really the outcome of the inception is a backlog. 0:04:43

\textbf{Interviewee:} So you've got some product owner, who culls the outcome of that into a prioritized list of stories, each one of those describes at tiny interaction between this person and the software that we're building.  Ideally, there's variations on that but then, so that's what an individual user story is.  0:04:58

\textbf{Interviewee:}   And then we go to kick off.  So then, we have our first iteration planning meeting where we take as input the prioritized backlog and for each story we go through them.  And by then, the product owner has got them into a point where I call it readied, they've met the definition of ready, which is they have clear acceptance criteria.  There might even be, a pre-IPM where that work is done, as well as, so this is the input is the backlog and the output is the backlog in the pre-IPM.  There are two things that happen, one is that we get clarity early on what the requirement is, and the other is that we get technical input to help with that prioritization and viability, like, ‘ok this may seem simple, like on the face of it', but actually, there's a whole of things that has to happen to make that work, etc. So we surface some of that, the pre-IPM includes somebody from the product side of the house and someone from development, so product owner, development.  Sometimes depending on those, the story can get more complicated, the more requirements, with the greater the variety or the more exotic the product is.  So you might even need someone in from design who is kind of help guiding how this should unfold and the interactions between things because there's all kinds of decisions from that end.  I Imagine, although, we haven't done this yet, in an enterprise client that you might even have somebody from like architecture there, business architecture.  The point of this conversation is to get all of the perspectives that need to be folded into the prioritization and the validation of, that the stories are legit.  0:07:11

\textbf{Todd:}  	Yes.  0:07:12

\textbf{Interviewee:}  	So then that goes into IPM, and this is a straight-forward roll through - the from top to bottom - the backlog, and we go over each story.  We read out title, we read through the acceptance criteria and then the team points each of the stories, the purpose of that is to surface complexity and to get a general, like, understanding, like, common understanding of what this thing actually is, what it is, what's gonna be involved to building it.  0:07:52

\textbf{Todd:}  	Yes.  0:07:53

\textbf{Interviewee:}   	We don't give in to like, we try not to give in to implementation details.  Sometimes, we have to dip down for a second to like verify that we're all speaking the same language, that we're all really envisioning the same kind of thing.  But that usually is surfaced by like, \quotes{I pointed at one and you pointed at five,} and then I would like \quotes{what?}  So then that hopefully prompts a conversation.  If we all said three, it's good enough for now we move on, if we all think it's about the same thing and the product owner doesn't lose their mind hearing that number, then we're all kind of okay.  Then the output of the IPM is individual points on the stories, and we typically go for some amount of horizon so the minimum that I feel comfortable walking out with is at least for the week, so we don't have to, like, break the flow of getting work done before the next IPM because we're gonna do this once a week, these IPMs we're gonna do this once every week. But ideally, a little bit more runway so that we mitigate against that the team haven't jumped out of the flow and also to, like, help the product owner have some room to sort of steer.  If they only have so many points in the stories, then they have to kind of pay the price of reprioritizing. So that's IPM.  Haven't have written a line of code yet.   So then, after IPM, we get working. So out of the backlog, a pair picks up a story and so then that pair looks at the story...  How deep do you want me to go, because I can go all the way down to like testing and stuff.  0:09:48

\textbf{Todd:}  	Sure.  0:09:49

\textbf{Interviewee:}	Ok.  0:09:50

\textbf{Todd:}	This is your diagram.  There's no wrong answer.  0:09:54

\textbf{Interviewee:}  	Alright. Ok.  So, the pair looks at the acceptance criteria and they say, \quotes{hmmm.... what test do I need to write that will help?  Or a set of test that will help if those tests ran green?  I feel really confident that we've met the spirit of that story.}  So they start there.  Usually at pivotal will do typically outside in so that means if we write something that looks like an acceptance test, something that describes very closely what this person with the persona experiences both in what they do, and what they see back from the software.  So we write the starts, we start with a very low fidelity version of that. It's not gonna describe the whole interaction, it might describe the smallest piece of we could possibly articulate; so we start there.  And then we run that test and it fails.  Then we say, \quotes{hmmm, ok, so we're in this architecture, what's the next part that we need to build in order to begin to meet those needs?}  And we work our way down the architecture.  In a typical web app, there's something that's displaying, an HTML page, and then there's something that is probably orchestrating the generation of that HTML, like a controller and usually we try and separate our concerns. We think about these things as we work our way down driven by trying to meet just this one acceptance test.  0:11:26

\textbf{Interviewee:} So along the way what we do is we write individual unit test for the components that have interesting behavior.  The controller does have interesting behavior.  It takes in some input and it makes some decision about what should be the output, what should be the resulting HTML.  And that controller also interacts with other collaborators.  So we have tests that say, are you properly handing off these parameters?  So these really fine-grain unit tests. But the key is that these things are, each time we write these tests, we're setting an expectation on that little tiny piece of the system, in the same way that our acceptance criteria are setting an expectation on the software, and our user stories are setting expectations on the feature .  So forth, all the way back up. We're trying to be needs-based  all the way down as we do this; that's probably good enough.  The details of how that happens varies wildly and even like within this, there are different schools of thought about how that happens.  There's people who believe in writing units for every little thing and people who say \quotes{no, you can set certain bullwarks} and write tests around the bullwarks.  Let everything sort of float in between.  0:12:43

\textbf{Todd:}  	It's a whole like, Chicago versus London.  Like, on how much mocking and doubles do we have?  0:12:49

\textbf{Interviewee:}  	That's right.  0:12:50

\textbf{Todd:}  	How much more integration of the unit test level do we have.  Are we really testing real things or not.  0:12:55

\textbf{Interviewee:}  	Yes, exactly.  0:12:56

\textbf{Todd:}  	Yeah.  0:12:57

\textbf{Interviewee:}  	Exactly.  So and the other part, too, is I think a factor in there in the choices that I make about these things, have to do with who the team is.  On like what their level of experience with test driven development is, and if it's really, really low, then I'll tend to want to have them write many more unit tests.  And I know that that is more fine grain tests, I know that that's harder at first but then they more quickly ramp up because they have more guard rails around them to guide them along their way, as they're learning the knack, the feel, the tacit experience of being test-driven.  Then as they start to become more comfortable, we can back off a little bit on some of these unit tests; especially if they're not catching anything, so test have a shelf life and if they haven't caught a defect, then there's a question of whether or not it's actually carrying its way.  So that can be a way of starting to prune part our of test system.  0:14:10

\textbf{Todd:}  	There's a bunch of people who are concerned about declarative tests or tests that test declarative code.  0:14:16

\textbf{Interviewee:}  	Ooh. Yeah. 0:14:17

\textbf{Todd:}  	Do you worry about that when you're thinking about this?  Difference is Rails has a lot of configuration and it's like why test the configuration when we already know that Rails works.  0:14:26

\textbf{Interviewee:}  	Right.  0:14:27

\textbf{Todd:}  	And that's through another framework to like... 0:14:30

\textbf{Interviewee:}  	Yeah.   0:14:31

\textbf{Todd:}  	Or does that not really end, you just like, \quotes{hey, let's just test everything?}  0:14:34

\textbf{Interviewee:}	 No, it does matter.  And I don't know that I have like a pat answer.  I think I'll probably flip flop depending on my mood and who knows what else.  I think if it's something where it is like a one-time configuration, so some of the factors are “how often are we gonna touch this;”  “how bad is it if we get wrong?” If it's like important plumbing that could get broken later on or something like that.  Coz you can do bindings, right.  Like bindings are kind of declarative piece so you tests those.  So I'd start off probably testing that kind of stuff.  But I'm not gonna test to see that configuration params are correct.  0:15:36

\textbf{Todd:}  	Because presumably it won't start up if they are wrong.   0:15:40

\textbf{Interviewee:}  	Exactly.  0:15:41

\textbf{Todd:}  	I interrupted you.  Did you finish with your diagram to the very end of the story?  0:15:45

\textbf{Interviewee:}  	So we get through the story, and then there is more to this process here.  As we go along through this development, some things feel easy and flow well and some things don't; and the things that don't, they go off on, what I call a pain's list, and I learned this from George Dean.  The idea is that let's enumerate things, let's just quickly note things that are challenging.  We may even have to work around them, like, \quotes{ah I want to take this certain tack but I can't,} so we note it.  And then, what I'd like to be able to do is get to the point where I can click finish on my story in Tracker.  But before I do that, when I feel like I've met all my acceptance criteria.  All my acceptance tests have met the acceptance criteria, then we'll stop as a pair, we'll quickly triage this list, some of the items just suck and we're not gonna do anything about it right now.  It's probably just, it's something that, it's not worth calling out as a chore yet. Or maybe it's too big, too; or maybe now after having gotten through the story, maybe we've changed our mind, maybe we have different perspectives, something like this.  We'll do one of three things with it, one we'll either address it right now, like \quotes{you know what, this is a two-point story and my god we flew through this, it took us an hour or two. Let's say we've been tracking a little bit longer for those kinds of stories.}  I feel like I have a little bit of buffer in terms of raising the quality dial a little bit so we'll address a pain.  What we're trying to do is hit the thing that's gonna be the least of effort for the most bang. We're looking at things, there often things like, it's all about maintainability, making the software more of an asset than an expense.  So readability, where there's something that's different for no good reason, bringing that back in line, all those kinds of things, so we'll address it - that's one option.  Two, we say, we're not gonna address it now but it's pretty important; it's one of those things where this is really gonna bite us if we don't do something about this or it's gonna create a ton of back flow where we will be blocked a whole bunch of stories if we don't address this soon.  So we'll mark it as a chore.  Or three we say, \quotes{yuck, don't like this.  It isn't necessarily gonna bite us right now, and if it really is painful, we're gonna discover it again}, so we delete it.  We try to empty our pains list and that's the last step and then we go and we click finish.   0:18:49

\textbf{Interviewee:}   As a part of clicking finish, either, and I've seen this in many different ways, either it's automatically hooked up from our version control system in our CI, we have a CI box, some CI setup that's listening to the same repository that our code is going into and it kicks off a build and when we commit with a certain message, it will mark it as, it will click the finish button for us effectively, finishes and then the tracker story number.  0:19:33

\textbf{Interviewee:}   Meanwhile, and this next step might have happen as a part of this whole development process. As long as everything is running green, I could actually commit and I'm not delivering a feature that couldn't be deployed, so maybe partial but if it's not seen, then it is ok.  So anyway, my point being that, we have a continuous integration server that acts like the nth + 1 developer on the team.  So it goes and it fetches the code just like the developer would.  Actually, the initial version of this was called integration station, where it was literally a box at the end of, like an end cap on a team's desk, and you'd get done, and you'd walk over, you‘d manually do this process.  And now it's automated.  The point being that we're demonstrating that the code can be built from scratch reliably from master.  Which means it's included in everybody's work up to that point and then all test runs. 0:20:39

\textbf{Todd:}  	Nice.  0:20:40

\textbf{Interviewee:}  	Ok so when we click finish, we see the CI build go green, then this varies also wildly.  We wanna deploy to some kind of acceptance area; so maybe it's called UAT somewhere, sometimes it's called staging, whatever it is.  So it's a non-production environment that now, the product owner can go and vet that feature against the story, thinking about this person and getting their job done.  So that can either be accepted or rejected; if it's rejected, then it goes back into the backlog.  And then, I didn't really make enough room for this.  0:21:34

\textbf{Todd:}  	If you need to use a second sheet, you can.  0:21:36

\textbf{Interviewee:}  	Ok, Cool.  So let's go like this.  So that's where the acceptance happens.  And then time passes, so we go about for a week and then we get together as a team and we dead reckon, at the end of the IPM, it's almost sort of like we all sort of have an idea about where we're going and what we think we're gonna accomplish.  0:22:02

\textbf{Todd:}  	Yes.  0:22:03

\textbf{Interviewee:}  	And then wind and weather, like, affect our actual location where we end up, so we need to, as a group, see where we're at, both in terms of progress we've made in development of the product and what we've learned along the way in terms of working with each other, and the technology that we're using, and what we've learned about the product and all these stuff.  It's navigating in all these dimensions so that's what the retrospective is about.  And so there, we step out of getting things done mode and trying to into a mode of dis-identifying that and reflecting on the process.  And we have kind of a pat way that we sort of do it at pivotal, there's a lot of different ways to retrospect.  But we focus typically on emotion-based sense-making.  So we say, what makes you happy, what leaves you sort of neutral, what might make you angry or frustrated or whatever.  That's a terrific way of getting an idea about, like our emotions are a really good way of developing situation awareness so we're gonna have one way of doing it, it's probably the best way that I would know of.  So we just throw up a happy face, a middle face and a frowning face up the board. We have people register anything. You could just be having a catharsis in front of the group, that's fine.  The point is to try and make some sense of what just happened just this last week.  And ideally, we come up with action items; and these could be in the form of what actual people do, so \quotes{oh, so and so needs to add a chore,} or we need to change this like about how we, the time that we do our stand-up or something like that, or they could be team agreements, so we clarify what the definition of done is.  Don't click finish until you see it, you smoke it yourself in the staging area before you hand it off.  Or please include instructions in the user story about how the product owner should accept this.  Some sample data or whatever it is.  We come to try, we looking to continuously try to improve as a team.  And that's the moment where that happens.  And then ideally, you kick the whole process off by jumping back into an IPM away we go again. Ideally, along the way, in parallel, this collective, this little mini team here is keeping the backlog groomed, looking ahead trying to think about, again, dependencies to both end.  What's important, what needs to get done in order to make a story successful and then rinse, repeat.  0:25:00

\textbf{Todd:}  	I think I got it.  Anything more that we need to add to this?  0:25:04

\textbf{Interviewee:}  	Let's see if there is any major pieces missing.  Now, I said happy path.  0:25:16

\textbf{Todd:}  	Yes.  0:25:17

\textbf{Interviewee:} 	So there is one fork off that can happen, way up in the front, you know what I was saying, the OD or sales or someone could be vetting the customer, now could very well be that they just have no idea, they're totally barking up the wrong tree, or they're asking us to do stuff that are just so out of what we do, that we direct them to somebody else or we try to give them some feedback, that like, you're not ready for this.  Or it's not a good fit.  Another possibility is they're actually kinda close, and if they can get a little bit of help of defining what it is that they want, then we can maybe get them there, that maybe we can get them in a room and actually do a valuable scoping.   And so what we do is offer them a Design and Framing.  And this is where we take them through a workshop where we help them unearth what they do know about their market and their product and start to explore their business model a little bit, and try and talk about, get them to think out loud about what they do and don't know about that. There's a sort of like exploration phase of it.  Ideally, what we're doing is getting all this complexity out of their heads on to the table and we then try and help them converge on, prioritize and focus, force them through various types of focusing exercises to really get down to like, what is the most important part of their business right now.  0:27:02

\textbf{Todd:}   Yes.  0:27:03

\textbf{Interviewee:}  	Get clear on what you know and what you're speculating on and what you need to know.  Like get into like, what do you need to learn kind of thing as supposed to how much money you need to make kind of focus.  0:27:16

\textbf{Todd:}  	So if I understood you right, it sounds like the DNF was mostly for the unhappy path.   0:27:22

\textbf{Interviewee:}  	It is \ldots  0:27:23

\textbf{Todd:}  	Could you see people in the happy path needing that?  Maybe I misheard you.  0:27:28

\textbf{Interviewee:}  	Yeah, yes.  0:27:33

\textbf{Todd:} 	Maybe think about your own projects.  How many of them do you think should have a DNF at the beginning of them?  0:27:40

\textbf{Interviewee:} 	So let me see.  So we did one for Sundance, it was one of the first.  Yeah, so and as an anchor, I haven't been involved in actually any of them that's been on the project.  So that's one big gap, like, in the same way that we should have a balanced team sitting in the pre-IPM, early up in this phase, it's I feel like we could do better in terms of getting all the voices in the room.  Or at least having some continuity of context.  0:28:25

\textbf{Todd:}	  Having engineering involved in the DNF sounds beneficial.  0:28:30

\textbf{Interviewee:}  	Yeah, yeah. And it's not just that they're participating, there's context that happens.  In the same way that part of the value of pairing is not just what design decisions were made, but those that were discarded and the reason why they were discarded, like why we didn't go down this path.  There's lots of context to be gained, understanding a more refined meta model by participating, being directly involved and hearing exactly how people are expressing themselves and seeing the connections between the people and things in that level of discussion in that business model discussion.  0:29:10

\textbf{Todd:}  	Ok.  0:29:11

\textbf{Interviewee:}  	So, it's not just that there is somebody there with that expertise I think it's also that as we move into the execution, the more that the people who are doing the work can understand that, then they can make better choices in the trenches.  0:29:27

\textbf{Todd:}	I think for me I thought of the design and framing is, if you have a client that doesn't have a clear understanding of what they wanna build, they may think they do.  Or it's too big to do one engagement and it seems to me like a good pre-filter to the engineering cycle before you have expensive engineers on this thing writing code.  Just really make sure and validate that we're building the right thing.  And we actually understand the user that we're building it for.  0:29:51

\textbf{Interviewee:}	Right, yup.  0:29:53

\textbf{Todd:}  	So I'm really intrigued by the idea, is this a normal sequence or is this when we have, like, a client that's not a good fit and we're trying... you know what I mean?  0:30:00

\textbf{Interviewee:} 	The way that we've talked, the way that I've heard it framed, it's when the client is, like, one step away from being ready.  0:30:11

\textbf{Todd:}  	Interesting.  0:30:12

\textbf{Interviewee:} 	And what we're trying to do is help them get that additional level of clarity so that they can get into the room and scope so they can actually articulate software features so the rest of this process can run.  0:30:24

\textbf{Todd:} 	Now, just to clarify. For me when I think it ready, I think maybe when they have their own usability team, design team, they might have done all the things we would do, then they're for the ready. But listening to you, maybe ready means something different.  0:30:38

\textbf{Interviewee:}	So, at a bare minimum,  it's that they can articulate what features they want in their software and that they have some kind of plausible story for how investing in that is gonna help them yield their goal.  0:30:58

\textbf{Todd:}	If they have that they're ready?  0:31:02

\textbf{Interviewee:}	Yeah, I think that's the criteria we've been using to get there.  0:31:09

\textbf{Todd:} 	<interrupt> Okay, this is helpful for us.  0:31:10

\textbf{Interviewee:} 	We've been using to get there.  0:31:12

\textbf{Todd:}	Nice.  0:31:14

\textbf{Interviewee:}	Yeah, that's my understanding.  0:31:15

\textbf{Todd:}	Ok.  0:31:16

\textbf{Interviewee:}	Whether or not, that's actually, coz I'm not the OD, I'm not in sales, I'm not purview to those vetting decisions but I hear about them.  0:31:30

\textbf{Todd:}	So we're talking about exception paths. Are there other negative flows?  0:31:36

\textbf{Interviewee:}  	Let's see.  Well there's sort of some obvious stuff like contract negotiation, someone gets kicked out because, \quotes{oh my god, you're that expensive?}  Didn't you read the price sheet? Let's see there's that... Things can happen like where we inadvertently discover later on the process, so like \quotes{oh wow! We are so not ready.}  That can happen.  I can't give you a specific example where we went into an inception and had to abort; but I've been in a couple of scopings personally where we kinda thought things are sort of new and then we got into it and as I started talking about it in more depth, it was like, \quotes{oooh yeah.}  For one reason or another, they didn't appreciate yeah, so there's that.  0:32:49

\textbf{Interviewee:} 	All kinds of things can happen in a pre-IPM, you could like clear the decks, like \quotes{oh my god, we're working on the wrong thing when you should totally work on this thing.}  That can happen.  It doesn't necessarily go linearly, well IPMs can go off the rails when we haven't done a good job in clarifying and, like, getting the stories to meet the definition of ready.  There's all kinds of back flows that can really kind of like this is a crucial piece like in terms of, like, actual production itself.  You can get back flows anytime, there is an invalid assumption or worse, un-surfaced assumption. And those can be off effectively kick backed into a pre-IPM state, if you will, mark the stories as blocked and move on.  In the software development experience itself, like you're wrong 99% of the time, and then until you're right for the 1% and then you run into the next wall.  There's constant back flows. 0:33:54

\textbf{Todd:}  	Right.  0:33:55

\textbf{Interviewee:} 	See our builds will fail every so often even if you're super diligent, that just happens; and there's to the extent that configuration is not automated and some things like migration or change in dependencies or things like that.  Things can fail in acceptance as well and those are backflows; but they're not horrible exception cases.  0:34:32

\textbf{Todd:}	Fantastic.   0:34:35

\textbf{Interviewee:}	Yeah.  0:34:36

\textbf{Todd:}	This is very similar to the way I think of the way we develop software.  It's nice to see you draw it out this way, so thank you.  I have so many questions I wanna ask.  I can't contain myself.  I'll resist all my questions, and I think this is a good one for us right now.  When you think of the current project you are on, what are some of the challenges that you're facing?  Either as an anchor or as a software developer or as a pivot for the project?  What makes this project special or unique?  What's the pain points that you're feeling?  0:35:15

\textbf{Interviewee:}  	Ok, right now, our biggest pain point is that we're in a halfway state type engagement with my current project.  What started out as a unified team, had to split into 2 teams because someone with a lot of influence came in to lead, lead part of the team.  He does not, at all, see the value of an iterative incremental approach to software and just kind of like down the list.  Everything we do, like, it just doesn't compute for him.  And he's been very successful from his perspective at doing it the way he does it.  And so we had to split off a separate team that is working in an XP way and so we're doing development separately from this other team.  So the biggest pain point around this is the clarity about how much are we… coz we're consulting, we don't just develop software, the whole point of why people come to us is to enable them to do it themselves.  0:36:42

\textbf{Todd:}  	But he doesn't want to do it the way we're doing it.  0:36:44

\textbf{Interviewee:}	Oh, no. and everyone else on the team could totally go with it.  0:36:49

\textbf{Todd:}  	Go with his way?  0:36:50

\textbf{Interviewee:}  	No, go with our way.  0:36:51

\textbf{Todd:}  	Oh, ok.  0:36:52

\textbf{Interviewee:}  	But he is so influential, politically, like organizationally, he gets to come in and totally dictate like, this is how it's going to be done.  So there are moments where, even the product owner, like, wants to...  He has seen, the product owner has been, lived through, been a product owner on a pivotal project – multiple.  And he says, "I know what it looks like, I know what that flow is, it's amazing, I want that.  But I can't do it because of this guy."  He is acting like the technical lead, he's a business architect.  At least he's coding, but it's this chief surgeon style, so  everybody has to surround him and he holds the whole architecture in his head and the rest are billings (?) and scribblings and he doles them out in terms of tasks.  So it's very much like centralized approach there. And so there's even this desire to like, \quotes{hey can you help us be more test-driven even if it's hard to do that in this platform.}  This guy says we have to do it in.  Where the team is programming in XSLTs and a little bit of Java modules around it, and he built up this huge universal business widget adapter that takes any possible XML message and can apply any number of XML transforms and can then output to interact with any kind of end point.  It has this event model engine.  The whole thing is just like a Rube Goldberg type, it's amazing.  They're getting it going and the thing is... I'm getting off track...   0:38:43

\textbf{Interviewee:}   The pain point is that there are these interactions along the way, that's in a retrospective or even moments like we had an IPM where they...  Somehow the product owner was able to convince this guy to at least stop using a spreadsheet to track tasks and make them chores in tracker. So at least there is some hope of having something that you can prioritize and move around, that there's some visibility outside of this spreadsheet that's updated regularly etc.  So we just naturally me and the guy I'm working with, kind of co-anchoring, we just naturally said, we took the first story and said ‘how do we know this is gonna be done?', because it was just a task.  So through, we rotated these things 90 degrees so that they have acceptance criteria, that they have outcome language around them, instead of output language.  That was like all these consulting that we would normally do; if you see a backlog gone sideways, like, that's what you do. But it's unclear whether or not that was actually part of our engagement, because, they're like a separate team.  The pivotal product manager was out for the week and so as co-anchors, we filled in for him.  0:40:04

\textbf{Todd:}  	Was that appreciated?  0:40:07

\textbf{Interviewee:}  	The product owner was, like, delighted.  And I think that we... 0:40:12

\textbf{Todd:}  	Feel free to open that thing (Kambucha bottle).  0:40:14

\textbf{Interviewee:}  	I'm just trying not to not make a mess.  I think I'm just gonna commit.  0:40:19

\textbf{Todd:}  	Yes, you did it.  0:40:20

\textbf{Interviewee:}  So he was delighted and there's actually a program manager that is overseeing all of the development here for Corelogic.  0:40:37

\textbf{Todd:}  	No, oh no. I'm putting 2 and 2 together. (This company is trying to adopt the entire pivotal way of working for their entire company, but there are clearly people there who do not embrace this approach to software development.)  0:40:50

\textbf{Interviewee:}  	Ok.   0:40:51

\textbf{Todd:}  I think this architect might feel threatened by the way we work.  It's just the sense of value, he would feel very...  0:41:09

\textbf{Interviewee:}  	Who knows what's going on in a person's head?  0:41:12

\textbf{Todd:}  	Then we have whole team and code ownership and all these things and if all of your value comes from being the chief surgeon then another way of working might be scary.  0:41:24

\textbf{Interviewee:}  	And yeah, it's, hard to know. I mean, how, exactly, is he responding to it?  It might have flipped  the bozo bit, you know, for him for all. He may have decided early on, well, this is all well and good for start-ups but I'm doing enterprise software here and you don't increment your way to high volume message processing.   This is industrial software, you architect this thing.  It may not be that he feels threatened, it may that be he firmly believes the way that he's approaching is the appropriate way to approach it and that because he hasn't had that experience of what it's like to navigate his way through a project, that he doesn't have faith that it can be done.  0:42:21

\textbf{Todd:}  	So how do you find yourself tailorizing this process?  If at all, to handle the situation? 0:42:30

\textbf{Interviewee:}  	First of all, hats off to Mr. Gerhard and Bearnek because they… so Mike Gerhard was the anchor before myself and Mike Bearnek came in to help out.  What they did was they helped define: this is your dance space, this is our dance space, this is what we are on the hook for, this is the process by which we're going to take your hundred page architecture document and find ways of getting in to a scoping so we can to the pivotal process.  So they helped create lots of clarity in terms of that. We will be in a world of hurt if it weren't for them.  And so we have this piece that we're developing, that's the web front end to this message processor thing, so the pieces that we have those interactions like what I was talking about was mostly like our PM is having to figure out how much of his time is he gonna spend as a glorified secretary  versus actually showing process. And it's every so often there's, so let's see, from my experience is... how are we doing on time?  0:43:46

\textbf{Todd:}  	Doing good.  0:43:48

\textbf{Interviewee:}  	So my experience around all this is there are a few people that are on that other team that periodically have an opportunity to cross over and they get to pair with us. And we just like put out the red welcome carpet, any moment you want, we can split our pair, like we're ready to go and every time they've come over, we've just accommodated them.   And we had a couple of really good pairing sessions.  But these are also like key technical folks on the implementation of that processor piece as well.  It's far and few between and they're just getting crumbs it terms of experiencing the whole flow.  0:44:33

\textbf{Todd:}  	Coz the whole company is trying to adopt to the way we're working.   And there's room for like iteration zero, where \quotes{hey, we're built enough systems that we know to architect a web app and IOS app and Android app.}  If this thing is a really brand new widget that no one's ever built before, I mean, you could do some architecting to understand what's going on but then you would wanna flow into an incremental building of this thing.  0:44:58

\textbf{Interviewee:}  	At the very least, even if it's like you have a target platform, thin slice develop it.  There are ways of doing it, even if, like you are not building custom software.  0:45:09

\textbf{Todd:}  	Somehow this individual was able to skirt around all these even the whole company is theoretically retooling itself. 0:45:14

\textbf{Interviewee:}  	At one point, the product owner on the client side, had mike code, like, developing in parallel, like another solution, and basically taking that thin-slice approach.  And the director of Core logic Labs said \quotes{Knock that off}, like, so, and I don't truly know, like what the conversations were. But that man sits at top of one organization that is a peer to the architectural organization, so now we're talking about who has influence up to the CIO.  0:45:57

\textbf{Todd:}	Interesting.  0:45:58

\textbf{Interviewee:}  	So the way I read the tea leaves?  It's that probably we pushed back and the marching orders came down, \quotes{you will use this product, you will use this architect.  Any questions?}  0:46:17

\textbf{Todd:}  	Yeah.  0:46:18

\textbf{Interviewee:}  	So...  0:46:19

\textbf{Todd:}  	So you are mostly using the same process.  But there is this tension points around this individual and this other team that you're still collaborating with, are there any integration point?  0:46:34

\textbf{Interviewee:}  	There is this integration point, like we're sharing a data store.   So there's a Mongo DB that they write to. And we're displaying the data that they write in there.  And that hasn't been too bad.  So yeah.  0:46:50

\textbf{Todd:}  	Can you think of another project that you have worked where you had to deal with an interesting client situation that forced you to tweak slightly the way we work.  0:46:59

\textbf{Interviewee:}  	Let's see.  I'm kind of rolling, I'm playing back the dates.  Most engagements, we're able to generally follow our process and the things that tend to vary around that.  I mean we don't make compromises on these larger pieces.   I don't think I've ever been on an engagement where we had to give up certain kind of testing for one reason or another.  Not use user stories or not do IPMs. It is a different story when we start to disengage and the client does what they wanna do.  I don't know how valuable this is gonna be but this is the closest thing I could think off.  I had a short stint on Grinder, and my mission was to join the team that's working on, there's an IOS app and a Java-based backend using Actor model.  Wow it is as if I blocked it out.   0:48:45

\textbf{Todd:}  	It's fine.  0:48:47

\textbf{Interviewee:}  	AKA and the guy that I was pairing with day-to-day, he saw the value in really trying to find ways of testing individual actors in isolation.  But the guy who is the tech lead of the group, who refuse to pair wouldn't come to our office, like he came once or maybe twice, just wouldn't, just totally didn't want to participate. He stayed in Hollywood even as the whole rest of the team was here and so one thing that we did that was just an adjunct to the process and adaptation was trying to improve communication with him and really trying to develop a trusted relationship with him. Because there was a lot of distrust.   So that was just more interpersonal stuff. We made sure we had a touch base with him every morning after our stand up.  And we would be as transparent as possible about what it was that we were doing.    And he would, you know, read us the riot act on this, that, or the other thing,  for things that we couldn't possibly have predicted or whatever.   He always had some kind of criticism or something like that in the beginning.  And then slowly over time, there was a little bit of figuring out the right boundaries for the relationship.  It was like the whole a lot of  interpersonal type of thing that went to that.  There was an add-on, real explicit add-on, like in that touch base.    And that happens normally in an engagement, like to me, that's sort of the heart of it.    Coz if we're pairing then it's a trust base relationship between the two of you and in very deep ways. In same thing with the product owner being able to make decisions, scheduling chores and things like that, etc.  Like there is that's all over the place but this one was really needed hands-on, give this guy lots of attention.  But we didn't break process, anyway it's just more like an add on there.  0:51:06

\textbf{Todd:}  Thanks for asking these questions. There's pivots we often feel like the one true way to do everything.  But in my experience, especially as an anchor, each project is different, it has different nuances,  I have to like understand just the players are different, the people are different.  So if a product manager, he's acting differently than the other product managers, I need to sort of shape the team differently, like, the relationship between the pm and the anchor depends upon the personalities, then in that dynamic might shape who does what and in what meeting.  It's just really interesting how, like what we do is very similar but there are sort of nuances to what we do in different patterns, different things that we can pull off in the playbooks to deal with different situations.  0:51:47

\textbf{Interviewee:}  	Cut playing to our strengths, generally speaking.  0:51:52

\textbf{Todd:}  	Maybe that's it.  0:51:54

\textbf{Interviewee:}	That's why I kind of see my role as an anchor, like one of my biggest thing is to trying to identify in everybody in the project like what their strengths are and play to those, elevate those and then where they have weaknesses, try and cover for them.  And ideally  help build them up if they have interest in that.  So somebody has the easiest one is like technical stuff.  Like somebody's weak in a particular area as we're picking pairing and something like that.   0:52:25

\textbf{Todd:}  	Your IPMs, what day of the week do you normally do them on?  0:52:29

\textbf{Interviewee:}  	Well, ideally, they're on a Monday, but the project that I'm on right now, I think, they're Wednesdays.  And we're retrospecting on Thursday. But I like to, I prefer a retrospective that is late on a Friday.  I'd do it with beer, and then IPM Monday early.  So it has the sense of we're setting goals and at the end of the week, we are letting off everything.  0:53:05

\textbf{Todd:}  	In the IPM, do you see yourself filling up a week's worth of stories for the backlog?  Or do you see your team kind of setting a target on where you want to be on a Friday?  0:53:17

\textbf{Interviewee:}  	It's more about keeping the backlog healthy than it is necessary setting a goal, you're right.  I have, kind of, a picture in my head about where I think we're going, which I think is about, to me it's more of about meeting the levels of situational awareness and focus.  And part of my job is to look further down the road.  0:53:44

\textbf{Todd:}  	Yes.  0:53:45

\textbf{Interviewee:}  	But I know that not everybody in the room is not necessarily thinking that way.  Some are, but others are just taking it, which is great, low stress, that's awesome.  You'll do better if you are able to focus.  And just sort of look up and \quotes{wow, look at all we've accomplished.}  0:54:02

\textbf{Todd:}	Moment of zen - tell me more.  0:54:06

\textbf{Interviewee:}  	What do you want to know?  0:54:08

\textbf{Todd:}  	I don't read every stand up email but I've seen a lot of moments of zen.  And I was curious on what inspires you to put that in there, how much effort it takes you to do it.  Do you try to do one a day?  Or does it show up?  Or...  0:54:29

\textbf{Interviewee:}  	So, I no longer do it.  I retired a couple of months back.  I started it during my first or second week that I was here at Pivotal.  And I honestly can't remember the thought process that brought me to the conclusion of, like, I should do this.  Being new to the white board concept and we were small office, and there was lots of space, if you will, in that timeslot.  There was sort of a vacuum almost to fill.  0:55:10

\textbf{Todd:}  	They pull in teeth to get anyone to say anything because no one's got something that's interesting.  0:55:13

\textbf{Interviewee:}  	Exactly, pretty much.  And so there will be no push back on having something to say but it did feel, I could tell you this.  One thing I was really clear about was I needed to tie the moment of zen back to our mission as an organization and so I became clear of that.  I just started doing it and then realized that I didn't share that thought process and a couple of weeks into it I did,   which was like, one of our core values is an empathetic connection.  Like, and that's the way in which we can understand our customers and their customers and each other on the team and then pairing that navigate that relationship.  My theory is that the better that I understand myself, the more that I can take responsibility for my baggage and be able to be aware of it, I can help set that aside so that I can be more fully present for whatever enactment I'm involved in, whether it's pairing or one of these other interactions.  And I could be more fully there with that person.  And that's what empathy is about, is making that connection, seeing their humanity and my humanity connected and through that doing work together.  So if we're prompted, in the kind of dripping drops of water, and the water jug is filled kind of way, with an invitation to reflect on “who am I?” And that's actually all of everything in there, the fundamental questions of who are you.  And if people can use that, if they choose to pick it up, use that to do some reflection and have a little bit of better idea of themselves.  Then as an organization, even if they don't, I didn't care at all whether or not anyone else did anything with it.  But if it helped by existing, by being present, it was saying this is our value.  This is one of our values, it is like a day-to-day reminder that we believe in empathy.  0:57:43

\textbf{Todd:}  	So, I'm kind of assuming that in the stand up at some point, you would sort of share a quote.  0:57:49

\textbf{Interviewee:}  	You haven't seen this.  0:57:51

\textbf{Todd:}  	I've only seen emails.   0:57:53

\textbf{Interviewee:}	Ok, so yeah, when it comes to the interestings, it got to a point where we all sort of had a routine around it.  So if anyone ever saw, and they did every single day that I was present.  I made sure I do one every single day that I was here.  I missed one once, I think. If they saw that it was up there, then they would do all the other interestings and make the moment of zen the lasting interesting.  They'd even say \quotes{any interesting before the moment of zen?} 0:58:23

\textbf{Todd:}  	It'll be the last of, the book-end of the stand-up.  0:58:24

\textbf{Interviewee:} 	There was always events but it kind of book end that.  It did have this sense of...  It was kinda cool because it did leave a little more space we can be in your head if the events didn't necessarily apply to you.  Yeah, because it was enough time where there are a number of people that would come up to me afterwards and say, this or that, or to have some kind of response.  0:58:54

\textbf{Todd:}  	Alright.  0:58:55

\textbf{Interviewee:}  	So they wait for that last one and then the moment of zen.  And so I refined my delivery of it so I'd say “today's moment of zen is brought to you by this person.”  And it has some kind of factoid about why you should hear from them, why you should even listen, what was it about Eleanor Roosevelt that was so special, what is it about Rom Dask or Allan Watts or Sherry Hoover or whomever the quote was from. So there's the quick blurb about what makes this person interesting and worthwhile listening to.  And then so I would read out the quote verbatim and then I would try and interpret “the explore”, I would try not to use the same words that were on the explore so that it would be more authentic, it would come across much more conversational.  0:59:48

\textbf{Todd:}  	In your voice even... 0:59:49

\textbf{Interviewee:}  	In my voice, and so.  I would usually start off with something like what a...  One way we can explore this today, one way we can see this in our life today would be such and such whatever it is.  1:00:04

\textbf{Todd:}  	Yeah.  1:00:05

\textbf{Interviewee:}  	And then I realized that a visual was really helpful, as well.  We'll just have a picture either of the person or even better if there was some kind of illustration that got to the heart of the concept and there are a couple of times where I got really lucky where the visual was actually the thing and the quote and the explore danced around it.  So they're all kind of like, played off of each other, and weren't redundant.  1:0029

\textbf{Todd:}  	Was there a sort of like a moment of reflection or was it moving straight into events?  It seems you said period.  1:00:38

\textbf{Interviewee:}  	Yeah, I would say, I pause and then I'd say good luck. And then we move on.  1:00:51

\textbf{Todd:}  	And you stopped doing it or you retired from it because…  1:00:58

\textbf{Interviewee:} 	It was in an emotional decision that I later rationalized.  1:01:01

\textbf{Todd:}  	Ok.  So what was the emotional decision then?  You can explain't it?  [1:01:05]

\textbf{Interviewee:}   	The emotional decision was, \quotes{oh I'm done.}  Like whatever it is, coz this was for me; this was very selfish.  This was like, I would do it in end. To answer your other question, I would do this every morning, there were a few weeks where I got up in front of it and was able to lay up a whole week. Often the ones that sort of a theme to them or a progression to them.  There's a number of where I was able to do that. But typically not, and so it would be something I do every day, it took me between 30 and 45 minutes a day, there was a lot of time.  And I would always start off with something and often end up somewhere else. So I would think that I a quote from, a bunch of different source be it from the books that I've read or from some online quotes, so then we would start searching around.  And then the hardest part was always the explore.  To me where I set the highest bar for myself was, in order for this to be value add, I really needed to do something to challenge people and it would even be better if there is a way that it could be, you often see it in a language like \quotes{in your pairing today, look at such and such;} so making it accessible.  It would be great if somebody else say something lofty and some great ism about life and then have something that's pragmatic and try to bring that into the ordinary.   1:02:37

\textbf{Todd:}	Nice, I think that's a good ending for an interview.  1:02:42

\textbf{Interviewee:}  	Beautiful.  1:02:43

\textbf{Todd:}  	Is there anything else on there is.  I need to stop so thank you very much.  1:02:49

% \section{2015-08-12 Product Manager Interview}

\begin{figure}[h]
\centering
\includegraphics[width=6.5in]{interviews/drawings/2015_08_12_pm.png}
\caption{\quotes{2015-08-12 Product Manager's drawing of software development process}}
\label{2015_08_12_pm}
\end{figure}


\textbf{Todd:} If you're open to it, could you, on that sheet of paper, draw out how you view the way we build software? This is completely open-ended. There's no wrong answers. And just take a stab at it.  0:00:21

\textbf{Interviewee:} So I'm drawing a line, it's like product on a continuum. We're gonna have vision here. We're gonna have working. Can you tell I'm a PM coz I'm writing in lines, boxes can't do? 0:00:45

\textbf{Todd:} I love it.  0:00:46

\textbf{Interviewee:} So let's see here. So let's do a couple of lines here. Alright, so let's say, each of these represents two weeks.  So we'll say that the first point, so how we build it at pivotal is that what you asked?  0:01:17

\textbf{Todd:} Yes.  0:01:18

\textbf{Interviewee:} So it starts with having clients.  0:01:20

\textbf{Todd:} How would you build it too?  0:01:21

\textbf{Interviewee:} Yeah, yeah. Totally. The reason I asked about Pivotal is 'cause we're doing consulting. So if you're thinking of product as a continuum, our clients come in in all these different ways of where they the insert. But it starts with a conversation, kind of like a qualifying conversation of like, 'hey, what do you wanna build?', and like 'what's the fidelity of your idea?' And from there, we have a good understanding of if they're at a stage where they can build something, or if they maybe really need to think about it further.  But once we realize ok, they're ready for Pivotal , we'll start with a Discovery and Framing and not every project gets Discovery and Framings, but in the LA office, a lot of them do.  We're trying to get like most projects getting some a semblance of Discovery and Framing.   0:02:04

\textbf{Todd:} When would you not do one?  0:02:07

\textbf{Interviewee:} So we won't do one... I think we can make the case really for all projects could do with a little bit of DNF right? So even if you're like, I know where to build from. So with FYI rather, but for a PM, I'm thinking, okay my role is to help the client understand who are we building for and then what we are building and when? So for the, who are we building for, I think all clients.  It'd be great if we could do some user research with them even if they are like, we did all these user research just to do a quick gut check, like hey this makes sense, that'll be great.   0:02:42

\textbf{Interviewee:} But I think a lot of the times when we don't do them, it tends to be convincing the client or if there's a budget concern.  So they have enough of a fidelity of understanding who their target user is and who the secondary users are and have a persona and are focused on stuff like kinda key factors like, ok, if you're saying you want to build an app for the millennials and for college-age students and for the ages 25-35, that's pretty vague.  We do get that a lot, but we're saying, ‘ok, it's for millennials, so what gender are we going for?' Or is it, ‘what are their behaviors like?'  There's a lot of questions that we can ask on this conversation. and these qualifying calls to say, to kind of fish out, to think if they are ready to work with Pivotal, or if they have enough information for DNF or not. So, does that answer?  I guess that was a very specific answer.  0:03:44

\textbf{Todd:} That was great.  0:03:46

\textbf{Todd:} I interrupted your flow.  0:03:48

\textbf{Interviewee:} No, no, this is fine.  So I'm of the opinion that I would love it if we could have some sort of research with all of our clients and users. Typically when we do a DNF, Discovery and Framing, we do that for 4 weeks on average, sometimes it's up to 6 weeks depending on how many users they have and how many people they want us to focus on.  But we typically really try to work on 2 to 3 target users being 1 primary user and maybe 2 users of the system that kind of insert in their day.  And then we've done a 2 week of design first or Discovery and Framing but really it's just 'let's talk to some users and validate some ideas.' But the whole goal of this Discovery and Framing process is to do a couple of weeks of talking to users, of about 2 weeks and that's when we're doing some of those exploratory interviews, kinda turn it into elicit narratives to understand what their behaviors and what their days are like.   0:04:51

\textbf{Interviewee:} And then from there, we can isolate what are some of their pain points and what are some of the frictions and inefficiencies and how are they capturing data and what are the tools that they use and who are the people they're talking to.  From there, I guess one thing I didn't mention which is important is to say, what are the product goals that our clients have?  And what are some of the assumptions that they have about their products?  About their users where their products can help solve their needs.  We go into these user research thing like, alright, here are some assumptions, hypothesis that we have, let's test them. Out of that output of user research is that we have this, we do some synthesis and analysis of all of the things they're talking about.  We record the things that we saw, so if they're in a cubicle and they have tons of printed out papers because what they do is they get stuff by mail and then they have to scan it in.  Those are the things that we're seeing that are part of their day that can affect them.   0:05:54

\textbf{Interviewee:} What did we hear, like things that they're telling us about their day, and things like we felt that they're telling us. But maybe there are some subtext and some nuance there saying everything's great but their faces are really strained and you can tell they're really frustrated and their posture has changed when they're talking about certain subjects. Kind of taking all of the things that recording from our user sessions and then coding it by what we saw, what we heard, what we felt and then finding themes. What are users talking about? Usually when we do research, we try to do 3 to 5 users, so that way we have a good cross-sample and in case there's any extreme people that we meet, it kind of helps us give a better analysis, better data sample. So we'll kind of call all the information that we have and we'll go to them, too.  So we'll go to their cubicles or wherever their workspaces are. So we want to get a sense of their environment. Cause there's so much contextual information there that you can't get just from having a phone call with someone.  0:07:02

\textbf{Todd:} Yes.  0:07:03

\textbf{Interviewee:} So, then, that's usually about a couple of weeks doing some researching.  As we're doing the researching, we're capturing all of our information on notepads and then we're doing what we call infinity mapping, affinity mapping.  We'll get one of those big phone cork white boards and we'll take a listen to the recordings of when we're doing user research or if we can't record just looking at our notes ‘coz our notes are like, you're almost writing verbatim what people are saying.  Coz you don't want to put all your analysis in there at this point, you just want to get them to talk and get it all in, and then we'll take each kind of idea and we'll put it on a post-it note and we'll have a bunch of post-it notes around and they'll be coded by what we saw, what we heard, what we felt, and then we'll start seeing themes around this post-it notes.   0:07:54

\textbf{Interviewee:} So there's this one user that said there is a lot of things around education and tools and timeline and whatever it is.  Then we started noticing trends, so we'll start taking those nuggets and we put them across this themes and then we'll do that to all our users; and then from there, we'll do another round of synthesis and the ideas are going to keep condensing.  So then we have a synthesis of all the people we talk to and what the overlaps are and the themes of what their day is like, and the behaviors they drawn during that day.  And then after that, we'll map out their day, like what are the tasks that they're doing, and then map out from the tasks all those insights rather not insights but nuggets of what we heard from them, that we kind of collectively called down and condensed and we'll put those against the tasks.    0:08:43

\textbf{Interviewee:} So when they need to schedule a user for an event, these are all the things they said about it, or these are the things we saw and the things that we felt.  So then what's cool about that is we do that for every target user and you can map them out.  So you say 'here are three target users, here are their days and here are all the intersections of their days'.   So you can tell visually, 'oh, you know what, when this person does something, you notice the next thing, like, these two people are affected.' And then you can start isolating pain points and inefficiencies and you get these really nice overlay of what the system, not the digital system but just the users in the workplace and what their days look like.  So when I'm mapping out the process, I'm thinking of conversations, that's a little talk bubble.  0:09:43

\textbf{Todd:} Ok, I like it.  0:09:44

\textbf{Interviewee:} And then I'm having doing some synthesis. Someone do a little post-it map.  0:09:51

\textbf{Todd:} Ooh, I see the post-its.  0:09:53

\textbf{Interviewee:} There you go.  And then from that, we say, 'ok, based on our research, this is what we say a persona or what this user looks like.'  We think of 'what do they need? What are the tasks in their day and what do we need.'  So we pull out insights from that. Like, this user really has trouble communicating with the other people on this team because they don't have the right communication tools setup or whatever that is. They need a better communication system.  Once we have these needs, and bits and insights of who they are and what they need, then we can say, "all right, how can our products solve these needs?" So then we say, 'we have some product ideas based on what their needs are, let's validate them.'   0:10:55

\textbf{Interviewee:} Then we'll go back and talk to the users. We'll drop some wireframes and say, "all right, based on what you guys had said, we feel that here's a quick prototype, clickable prototype" And we'll use invision.  We'll say, "Why don't you click around? What do you think about these things?" we'll do some user testing there.  And then from that information, we'll further validate or dis-validate our product ideas and we'll do another product evolution but kind of the output of this 4 week on average DNF cycle as that you'll have Wireframes and then you'll have some personas. You'll know who your end user is.  You'll have empathy drawn for your end user which is the whole goal. You'll have a problem that's been framed and validated.   0:11:38

\textbf{Interviewee:} So that way, it really de-risks development cause it's pretty great to come in to development and we know that when we're making product decisions, we can go back to this research.  We're like, "oh, if we're going to do this or this, like, what do they actually need? Like what was that they talked about that really indicated this is the right approach?" It also helps us speak a language that our users speak.  Which is really important for the development team but all the other stakeholders involved in the process.  0:12:05

\textbf{Interviewee:} Words are everything, right? So we want people to be kind of on the same communication levels of talking how their users would talk, so that way it helps us draw all that empathy throughout the entire development process coz you still need to draw on that, you know 6 weeks, 3 months, however long into the development cycle until you're releasing. Right to be able to have that insight of what they want. I think the cycle is... Let's just say that you've done some analysis and you've done some wireframing, and then after that, we'll talk to users again.  After that, we'll do another set of wireframes. We'll also do some persona mapping here as well as making these wireframes.  And then after we do that, we will create a feature list.  0:13:13

\textbf{Todd:} Yes.  0:13:14

\textbf{Interviewee:} So, we'll say, like, we're not gonna…  Like, what could the next you know 2 to 3 kind of epic areas. We'll say, let's give some insights here.  Like, maybe, we have 3 insights and 3 needs.  Let's do 3 feature ideas and how do these feature ideas map these needs?  So we write these feature ideas and we'll say, down in the documents and user needs this and we'll help them achieve their goals.  Business needs this and this will help them achieve their goals.  We're always aligning user business needs throughout the entire process. So then when we get into development, we'll just make a little terminal, I wish I had multiple colors.  0:14:00

\textbf{Todd:} Next time, you'll have to bring your can.  0:14:03

\textbf{Interviewee:} So, when you're getting into the computer, when you're getting into development, you're able to have a backlog that's been built.  You have the first couple of weeks of features.   0:14:17

\textbf{Todd:} Yes.  0:14:18

\textbf{Interviewee:} You have some ideas and you start building and then once you release something, which could be in a week or could be a couple of weeks depending on what you're doing.  But once you get into the first bit of business value working software, then you can go and you could talk to your users again and then you learn from them and then you continue on.  So at that point, what you're doing is you're building something and then you're measuring it.    0:14:45

\textbf{Todd:} Yes.   0:14:46

\textbf{Interviewee:} And then you're learning from it right?   0:14:48

\textbf{Todd:} Uh-huh. Yes.  0:14:50

\textbf{Interviewee:} And then you're going back to building, right? So that's what we're doing for this whole process. So the feedback loop is really important.  I think, when I'm thinking about the extent, like the depth of all these research, you don't need to have a DNF to do any of these research, right?  You don't have to sell that in, you can do...  So the training that our designers and our PMs have, and now we're trying to have our developers be exposed to it, as well.  To say, ‘ok, developer, you get a client project. Development starts tomorrow, you can still talk to some users.  You can still setup user testing.'  And you can do that throughout the entire process.  So it's almost like this whole upfront part, with like the DNF.  You're making little DNFs through it, right?  So whoops, it's really you're taking this and you're kind of doing like DNF cycle.   0:15:42

\textbf{Todd:} Yeah.   0:15:43

\textbf{Interviewee:} Whoops, and you're doing that here, and then you're going to do it again. So, like each week, you might be bi-weekly.   And that is ideal, right?  But sometimes you don't get the users?  So there's a lot of, kind of…  You have to be pretty scrappy with how you do this sometimes, you don't just get this nice set of users at your disposal, you know; but it's so important and I think that's something we do a good job with is convincing our clients, like, we wanna de-risk this for you so this is how we can do it.  I mean go, it makes sense.  It's very practical stuff.  It's not rocket science honestly.   0:16:20

\textbf{Todd:} Now are you currently PMing a project?   0:16:24

\textbf{Interviewee:} I am.   0:16:25

\textbf{Todd:} I kinda noticed that every project, like, by its grain, is different.  I mean there's some commonalities but each one has its own, it's like children, each one has their own personalities   0:16:33

\textbf{Interviewee:} Oh, my gosh, yeah totally.   0:16:35

\textbf{Todd:} So, your current project, when you think of its uniqueness, what are the challenges or pain points that make this particular engagement interesting?   0:16:47

\textbf{Interviewee:} Sure, you know I think that, I think this is truly an opportunity and cool but it's a huge challenge.  We're working on a recommendations engine, and this recommendation engine, what does that even mean?  It's like pretty broad language and so we are, and you know, the term algorithm gets thrown around a lot.  It's like, what does algorithm mean.  And we also have some data scientist in the office, so when we talk about algorithms to them versus our clients, different connotations of the word.   0:17:21

\textbf{Todd:} They talk about the heuristics as well?   0:17:22

\textbf{Interviewee:} Yeah, they do actually, that would be more when we're talking to our data scientist coz our clients aren't quite to that level of knowledge.  But I think that's one of the challenges is making sure, you know, as PMs, that we're helping them deliver value, and they're not getting overly complex before they need to.  They say, they came to pivotal because they want to build this recommendations engine.  We're like "cool, but before you build this recommendations engine, you need an application that has a certain amount of data and functionality before you could recommend anything, you need people, you need conditions, you need affinities, you need restrictions,” or whatever it is.   0:18:05

\textbf{Interviewee:} So it's been our process to get them to, you know, and they've been really supportive of it, but they excited right?  So it's like managing their expectations throughout this whole process to be like, "don't worry we'll get to it, but we need to get to all these other things."  And what we get to, maybe, a little different than what you expect and maybe we don't get to it all, right?  So we're actually at the end of the engagement and we're trying some really cool things that are little less sophisticated than they expect but fully meets their needs plus more.  So I think that one of the challenges with this project is for them to say, "Hey, ultimately, does your end user, are they able to do the thing that they need using this recommendations?  And do these recommendations make sense to them?”  So, yes, cool, that's success to me.  They can schedule, so it's a scheduling app and then, and the goal is really to schedule these users to events.   0:19:04

\textbf{Todd:} Yes.   0:19:05

\textbf{Interviewee:} And then this recommendation engine to make it easier for the scheduler so they don't have to think about why they're scheduling people but they're like "oh, well i can schedule quickly coz I don't have all the burden of detail of why this person can go here and this person can't".  Coz it's very intensive type of scheduling that they're doing, that's all based on this…   0:19:24

\textbf{Todd:} A lot of constraints?   0:19:25

\textbf{Interviewee:} Yeah, well you know I won't even say it's a lot of constraints actually coz I think that actually one of the challenges is that I feel like we almost like spending a lot of time on the edge case right now, that the bulk of the app that don't necessarily need this.  There's a little bit of proximity and some stuff that makes it easier. So I would say that recommendations engine I say in air quotes like 101.  We hit and they can do pretty early on in this application but we some of the other stuff they're… they have, they just manually take this historical information that the schedulers has and just adapt it into the system or at the system; but then they also need to see how the schedulers use it before adding more complexities, right?  Because I think they have data in the system that they can add to this recommendations engine but we don't know if that is really what they need. We don't know what's the right data that they're looking for this?  So I think they want to put a lot of work into something they are not ready for, but we strike that balance.  0:20:29

\textbf{Todd:} So given the challenges and the uniqueness of this project, have you found yourself tethering or adjusting the way we work to achieve those goals?  0:20:42

\textbf{Interviewee:} I think that.  So, like, agile it's like a set of different tools, right? So I don't think there's one way to do any of the things that we're doing.  I think there's a baseline of process that we certainly need to adhere to.  Like we always have a planning meeting, we always have a retrospective, we always have a stand-up.  I think that's helpful for the communications side coz most problems are people problem, so it's nice to have a good strong communication system in terms of like tools that we use to do, to make product decisions, to develop wireframes.  0:21:20

\textbf{Todd:} Trying to do a...  Like you do have a baseline that we all use.  I'm trying to find the, what are the little recipes that you pull off to spice up the soup we're making.   0:21:32

\textbf{Interviewee:} Yeah, I'm sorry i already forgot what specifically you asked for that question?   0:21:37

\textbf{Todd:} Oh, I guess this is pretty open-ended but curious like as a PM, do you have these tools that you just like almost like playbook items that you could go and reach for?  I'm kind of curious in the particular project which things were you reaching for than maybe having reached for in other engagements.   0:21:56

\textbf{Interviewee:} Right, right.  I think for this one, so this one I actually came in toward the end of the engagement coz I was in a different project and I'm not fully working on this project.  So I have a couple of things I was billed for which is a very unique thing at Labs that doesn't typically happen. So I've been doing much more like enablement teaching on like how I make product decisions, how I cut down on really thick backlog.  I think for we're using data science, so that's something unique for this engagement.  We've been pairing with other data scientists and I've been doing a little more like kind of high level conversations, like really our engineers are working with data scientists.  But I haven't really done anything I think that's been super kind of unique that I wouldn't do in another project.   0:22:55

\textbf{Interviewee:} The recommendation stuff is different, I haven't worked on a project with recommendation stuff, so there is been a bit more like a designer who has worked on those projects a while ago set up this really nice like set of equations that have to do with decayed log and some other kind of basic loose algorithm need, just some equations for waiting.  So there's been some additional research I've done on recommendations and, like, kind of spending a little bit more time with the developers thinking through how we would do certain weighing and sorting but I wouldn't say that there's anything that has been so unique to this project that I haven't applied to another project.   0:23:34

\textbf{Todd:} When you described enablement teaching a moment ago, is that for the client's PM or who are you enabling?   0:23:41

\textbf{Interviewee:} Oh, the clients.  So the founder, the CEO is our PM for this project.  So I've been working with him just making sure that there's the basics of, this is how you build a story.  Like using that mnemonic of invest rate, like, it's the right size and the right value and have the right description and then there's, you know, kind of, worthy end of the engagement right now.  So you wanna do all these things and you think critical for your MVP and for going live is this, this and this. How can we think about search instead of having this isn't specific for this.  But an idea would be, like, instead of the doing a search function, which can be, like, take all the time that you give it, maybe to control F on the keyboard and then you can see, you could search that way.   So kind of trying to figure out some like ways where he can still get the value he needs, where he doesn't need to add so much complexity to it, you know.  Like he wants to do this kind of write up kind of HR function. I'm like, "well you have email right now", I'm kind of like, "how many write-ups doing per day?", like "can you capture this on Google drive spreadsheets right now and track the write ups for these people that way instead of putting in these functionality", you know,  0:24:58

\textbf{Todd:} Into the tool, right?   0:24:59

\textbf{Interviewee:} Into the tool.   0:25:00

\textbf{Todd:} Yeah, it's interesting coz the idea doesn't... I've worked with the founder on the recommendations engine with some graduate students and it's just interesting how I picture he had some…  He kind of want to think things will work in a certain way and wasn't always clear, like, if his solution would solve his problems.  And so I feel, like, as a pm, I kind of wondering if you were sort of massaging, make him to, like, re-think his solutions and then understand his problems first?   0:25:29

\textbf{Interviewee:} Every day.   0:25:30

\textbf{Todd:} Ok.   0:25:31

\textbf{Interviewee:} Oh, every conversation.   0:25:32

\textbf{Todd:} And how does that work and do you like facilitate that?   0:25:34

\textbf{Interviewee:} Well, I think a lot of it is I think about, like, ok, again this kind of goes back to like what are your business goals.  Like, you have a lot of business goals but what are your top 3 business goals? Who are your top 3 users and what are their needs?  So when you're making decisions and if it doesn't align with these, or if it kind of align with this, it's really easy to be like, well, this is your budget, this is how many hours you have left, what's more important?  Ultimately there are other ways of thinking about it. So there's a lot of, use some facilitation techniques like, I'm sure you've heard of like a Dump \& Sort or 2x2, right?   0:26:15

\textbf{Todd:}  Oh, 2x2 yup.   0:26:16

\textbf{Interviewee:} Yup. Yup. So there's things that, like, what you'll do when it really comes to \quotes{I wanna do everything.}  So there's a point in this project not so long ago where the development team didn't see eye to eye on this path because this path for now we kind of getting into the heavier recommendations stuff.  And there's a lot of opinions on how this should be done.  So we did a workshop where we did a couple of rounds of Dump \& Sort and 2x2s, so how we did it was there was purposely a break in between with lunch.  The idea was like alright the question I asked was, "what do you think the path is for what we're trying to achieve" or something like that.  It was more specific it was like, "what do you think we should be doing?", like, let's do a Dump \& Sort on that and so everyone did it and we have variety of ideas and we then we put it on a 2x2 on complexity and value I think, value on development of complexity in value.  0:27:17

\textbf{Interviewee:} So coz we're thinking like time, and then like impact. So we did that and we got everyone closer to the same page at the end of like everyone did agree that this is the first thing that we should be working on now and this is the second thing.  And then we said, \quotes{ok, cool,} knowing all that, what are some of the technical implementation details coming from that.  So we did Dump & Sort and that we prioritized again.  We didn't do a 2x2 at that point.  We just kind of prioritized some ideas, then we had lunch, we came back and I asked the same first question of, so what do you think this same path forward was?  0:27:51

\textbf{Interviewee:} And it was interesting because my hypothesis is that they'd be more aligned but they weren't. They actually were, they were closer but what happened was after we did that, it was just the Dump \& Sort, we didn't do another 2x2, but then we said, \quotes{ok, so what are, let's talk about this.} What came out of it really quickly, because we already framed all these conversation was this idea was like, "oh, what if we do it this way?', and someone was like, "oh, that's great." so everyone had consensus within maybe 15 minutes after we had this like round 2 of asking the same question, people looking like they were kind of divided.  I think it was because people felt heard that there is another opportunity to talk through some of their thinking, that they were more amenable to other ideas and we're able to get to that open creative space.  It was really cool, but I was not expecting that we'd come back and everyone would be like so diverse again in their thinking.  It's interesting, so I guess that was different.   0:28:51

\textbf{Todd:} What would you do with the client that refuses to prioritize?  They would refuse if you brought them down to the table and say, \quotes{let's do a Dump \& Sort}, and they say, \quotes{No, we won't do that. Everything is of equal priority.}   0:29:04

\textbf{Interviewee:} Oh, so I've done that a lot and I've never had someone say no.  I don't know.  We just keep working on it in different ways.  So I think like...   0:29:14

\textbf{Todd:} It all has to be done by this date.   0:29:16

\textbf{Interviewee:} You know the date-time too, that's definitely come up, but I've never had someone who's been that absolutely no, like, I won't play because there here to work with us and I think they understand they're spending quite a bit of money working with us; even the ones who might not have that true number in their head, they get that sense of "I'm here to make something work". I think that people come in here for the most part, I haven't witnessed people that have been so 'I'm closed off, I don't wanna do this'.  But again it's kind of contextual, so it's easy for me to create like scenarios or ideas once I know what the product is.  So I'll tend to say, ok, because there's a lot of, like, writer's block or prioritization block.  Part of our goal is to be facilitators and to pull out ideas and say, like, "ok, cool.  You want these 4 things to be done by this timeline, so let's just, lets' focus on one.  What do you think the number 1 is, let's start somewhere”.  And them like, "it's all equal”, but we'll say this", and then we're like, "ok, next one, is this one more important or less important than that one?", "Oh, it's just as important", and then alright, so let's dig into to it.  \quotes{So what makes this more important?  How does it achieve?} I might start writing "what are the goals of the product here?" and then "who are your users?" and then I can even say ok.  And then I'll prioritize that.  Like, some will have some ideas and stuff like "here's all the things that's important and we'll try to do some prioritization of product goals. We can usually get somewhere when we prioritize something and then I can match whatever it is that we're talking about against that.   0:30:57

\textbf{Todd:} Yeah.   0:30:58

\textbf{Interviewee:} It's, like, how does this help you get to your timeline faster, right? How does this get someone to do this easier?  Ok, cool, so is this to be more complex than that, so then maybe timeline's the big thing here.  Then let's go with the thing that seems least complex.  There's a lot of that, like, we're kind of totally use that, like, you know, time-quality-budget triangle, the project manager triangle you always hear about.   0:31:25

\textbf{Interviewee:} So there's a little bit of that, but we have fun with it, too.  Like I try to, you know, people come in here with so many different reasons and perspectives on what they're doing and why they're here and so I try to keep it as light as possible, too and try to really decouple the emotions and the behaviors and kind of lean on the process more.  So it's not, \quotes{oh, your guys}  fault that you're not agreeing.  It's the process, we need to be able to facilitate an environment where we can talk about our ideas and prioritize them together, right?  So we really try to lean on these consulting tools and process these and whatnot to make it a little bit easier for people to get their minds out.  And the Dump \& Sorts nice because it's a really cheap and easy way of getting people just to get their ideas out and in the beginning they might have nothing or they might have the first three ideas really fast and like oh now i have nothing. and so they will say, great whatever you want, like, maybe an earthquake can shut down your business or you know whatever the prompt is, like, we'll think of some absurd things.  Write bananas if you need to or whatever.  You know what I mean?   0:32:41

\textbf{Todd:} From your perspective what is the purpose, what is your goal of the IPM?   0:32:47

\textbf{Interviewee:} Oh, so I think the goal of the IPM is to say, \quotes{Alright, where have we been and where are we going?} So this last week, I always start with the… I guess it's about getting people on the same page and with understanding the same vision and understanding what the goals are of the product for the week ahead.  I think it's important to say "alright, here are the big things, here's how our environment has changed based on what we've deployed and you know here's currently what's being worked on today since waiting for this meeting.  Are there any questions about it?  Is there anything that we need to know that's being worked on now that might change these future stories?" and then we go into each individual story and we say, "Alright, here's our treasure map. Let's talk about some of the hidden complexity from this."  It's not about, in my perspective, to have developers write out every single tasks and implementation details but it's kind of that space to have enough of a conversations so that you can estimate complexity as a team for a specific feature story, and that you can leave that meeting having a good understanding of \quotes{hey, here's how the next two weeks look like knowing that things change.}   0:34:09

\textbf{Todd:} Nice.  So you spend some time at the beginning of the IPM reviewing what happened recently. And it's like that 5, 10 minutes?   0:34:19

\textbf{Interviewee:} Yeah, well, so let's say the IPM submit in like an hour, so the first I might say, like, "hey, I'll have tracker up."  And I'll say, "here are the largest things that we delivered last week, just as you're thinking about complexity, here are some of the things that you might we learned that this integration point is behaves differently than we thought, and we have some research that we need to do."  But it's like a minute to a couple of minutes, we don't spend a lot of time, it's like a very quick recap.   0:35:00

\textbf{Todd:} Yes.   0:35:01

\textbf{Interviewee:} And then we'll say, "Alright.  Here's a couple more minutes and let's talk about what we're doing right now.  So is there any question on the stories that you're working on? So here are the stories that are in flight."  The planning meeting tends to be Mondays so there's, you know, depending on your team size, but if you have 2 to 3 pairs or 1 to 3 pairs, there's only a couple of stories that are being start or that are started.   0:35:22

\textbf{Todd:} That's true. Right.   0:35:23

\textbf{Interviewee:} And we don't go through like, all 6 stories from last week or whatever. It's just kind of quick context coz it's Monday and people have a weekend, just want to remember what we're all working on.  But then my agenda as a PM is I want to have all the stories pointed with complexity and that there's conversations had to enough fidelity where people have an understanding of what's going on and that they also know who to ask.  And if there's any blockers or dependencies that I need to know about so I can start unblocking things.  So if you think about it, if you're doing, like, what is it, 2 weeks, 1 try to go over 2 weeks of work so the current iteration, the next and if you have (this is for non-engineers) say 10 stories, right?  So you're trying to get through a 10 stories in a week, that's pretty high but whatever.  And so have 20 stories that you need to get through, and you want to spend a couple of minutes on each story.  Let's say you do 3 minutes on each story, so that's a full hour.  It's all about saying that pre-IPM or just like, not even if you're having a pre-IPM but all that you need to so much research and like good work and write the right stories and that goes as soon as possible.   0:36:36

\textbf{Todd:} There's a lot of people in the room.   0:36:37

\textbf{Interviewee:} Yeah, there is a lot of people in the room.  So it's about finding your rhythm but I think the goal is to make sure that you have a brief overview and that you can tell if it's taking longer than that few minutes per story.  The story's not ready and you immediately got to pull it in and put it in the Icebox.  That's on the PM to really be true to themselves and make sure they have that the ability, it's part of their facilitation to be like, ok, I think there need to be discussion so let's pull it out.   0:37:06

\textbf{Todd:} Yes, interesting. I was on a project where I've realized that we didn't really have iterations.   0:37:15

\textbf{Interviewee:} Oh, interesting.   0:37:16

\textbf{Todd:} We were actually very present with the work, we would just show up and do whatever's next on the backlog and the purpose of the IPM was basically to make sure there's a pointed the stories in the backlog.  And then it was really interesting moment for me.  Coz there's other agile techniques where you actually agree upon what you're going to do with your iteration like scrum.   0:37:40

\textbf{Interviewee:} Yup.  Yeah sure.   0:37:41

\textbf{Todd:} Kent Beck's Extreme Programming actually says you should do that, that you should actually commit to the work you are doing that week and I find that at least the projects I've been on were much more present that if a PM in the middle of the week says we have to change what we're doing, we just do that.  We alter what we're doing, we're very flexible and so for us on my teams, it doesn't really matter what day of the week we do the IPM on.  Sometimes, we do that on a Wednesday.  It's just that when the client is available, it really tends not to matter. It's just interesting.   0:38:17

\textbf{Interviewee:} I agree with that.  Like, I don't think we have this kind of Monday, like start of the week with the IPM, Friday retro but let's say you're working on 1 week iterations and your releases on Tuesdays, then when does your week start then.  Maybe you have your retros on Monday and you do your IPM on a Wednesday or something like that, Tuesdays your retros, however.  I think the goal is to be flexible as you're saying. You know, I think, it's all about your team rhythm and what you guys agree upon as this is the process that works well for you.  So I do think it's nice to have this kind of like you know, XP or your agile, whatever your kind of flavor of process that you're using and then you can adapt to it wherever you need.  You know I think 1 thing we need to watch out for consulting is our clients have different levels of backgrounds in this work.  And so we want to teach them like a good base level process so that they could then figure out kind of what their version of it is, and then also, the kind of speaking to them, no idea of an iteration.   0:39:27

\textbf{Interviewee:} I'm on board with that but you don't want it to be too jarring for people so when you're pulling stories in and out of the backlog too much within a week, you just want to make sure you're not doing too much context switching and that you're not, that work you're doing or whatever planning meeting that you're doing, it's noted and that you're kind of bringing that into your week. I don't know, I'm with you; I think it's great when you would need to be flexible, I would like that, I would actually prefer that. I just worry about the clients and their ability to work in those kinds of such flexible environments you know.  I do like that we give them this kind of, for their first engagement, we also have clients that come back for lots of different engagements, I don't know. I like it, I think and I've had teams where we've just been able to move things in and out but it's also been that the developers feel like there's not much thrashing and they're ok with it. I think it's easier for PM to do that but harder for an engineer to do that.  I think there's so much more thought going into the tables that they're working on into how they're structuring the project. I think it can be done in the right way but I'm always really conscious of that.   0:40:43

\textbf{Todd:} I think the saving grace is engineers don't want to work on non-pointed of stories so there is a gating factor. i mean, if i just dropped in 10 stories at the top of the backlog and they're not pointed, someone in the team's gonna say "hey, wait a minute, we need an IPM."   0:41:01

\textbf{Interviewee:} Well, I think it's scaled to, right?  Like a couple stories versus 10 stories that's different thing but...   0:41:08

\textbf{Todd:} It's like if I'm turning a steering ship a lot, there's gonna be push back and say, "wait a minute, we need to talk about this stuff."  I'm just not gonna pick up, and if I slip one in then you know that's ok.   0:41:20

\textbf{Interviewee:} I think it's like what's reasonable for your group, too. Like if you're finding that you can develop deliver value and you're testing with user and you're not doing so much big change that like...  I think you'll have to have healthy user testing process if you're gonna do big change really quickly as well.  Coz if you do all these change and then you never check with users frequently, then how well do you know that the change you're doing is right? Coz you can build in all these different directions and I'm all about like I do side on small experiments that you're testing, so when you're doing really big experiments or just more risks with what you're doing.   So I guess that's more of in terms of the pointing, you can have developers point, you don't have to be in an IPM to point, I guess.  I think that's a misconception like you can have a developer pick up a story and point together or point with 2 pairs if you want that kind of...  But it also depends on what you're working on, on what you're pointing, like a text change versus like an integration addition.  Those are very different conversations, too; so I think it is really contextual. That's what I would say.  But the trashing is what I'm always worried about.   0:42:34

\textbf{Todd:} Yeah, yeah.  That'd not be good.  Thank you.  You've been very delightful.  I think we could go on forever but I think I can't. So thank you so much.   0:42:45

% \section{2016-01-08 Interaction Designer Interview}

\begin{figure}[h]
\centering
\includegraphics[width=6.5in]{interviews/2016_01_08.png}
\caption{\quotes{2016-01-08 Interaction Designer's drawing of software development process}}
\label{2016_01_08}
\end{figure}

\textbf{Todd:} My first question for you is very open ended.  00:02

\textbf{Interviewee:} Okay.  00:03

\textbf{Todd:} There is no wrong answer.  I was hoping if you could draw how you feel about the product.  00:09

\textbf{Interviewee:} Okay.  This may get elaborate.  00:27

\textbf{Todd:} Fantastic.  00:29

\textbf{Interviewee:} This is my new pen so.  [Pause] Here we are.  I did sort of a story board-ish type of thing.  02:54

\textbf{Todd:} I love it.  02:55

\textbf{Interviewee:} So, do you want me to explain it?  02:58

\textbf{Todd:} Please.  02:58

\textbf{Interviewee:} Okay.  On one hand superficially, I'm happy because aesthetically, I think it's nice.  I think it's, when you compare it to some of the projects we do, it's been going on for so long and there's such a huge team of smart people.  I feel like we got so much done and it's complex and interesting and there's lot of thought that went into it and it's a really robust app but I'm worried that even though it's pretty and we built a lot of features and the technology is cool, not all of it is necessarily useful for end users and I didn't even give the user name anything because I don't even know that we're designing for the right person all the time with some of our features .  I don't know.  So, I'm worried there were will be more confusion in the marketplace than I would like there to be in a product that I've worked on.  03:57

\textbf{Todd:} Anything else?  04:01

\textbf{Interviewee:} In the drawing or in general?  04:05

\textbf{Todd:} Both.  04:06

\textbf{Interviewee:} Not so much in the drawing.  I guess the big question mark is just that I don't know, I think it's pretty usable but I'm worried there's going to be features like predictive for the users or going to be like what is this or why do I care or they'll have question marks around there's something really obvious to me like \ldots I would like you heard a lot of feedback, I want a light telling me if my oil is low and that's just not something we could do because of constraints on the technology I think.  So, I'm worried if people will look at it and say why is there all this stuff that I don't want and there's some stuff that maybe feels really obvious to some users that we haven't provided for one reason or another but overall, I still feel happy.  I think we created a solid product.  04:59

\textbf{Todd:} Good.  When you were describing your \quotes{you} picture, you used the word superficially.  I don't remember the exact word but something like given this superficially I feel happy about it and I was curious if there was like an under feeling of the product.  05:17

\textbf{Interviewee:} Yeah, currently to some degree, this is the under feeling.  The superficial part is a little bit as a designer is a normal person walking around, you feel like people look at it and like it's so beautiful and that might be the beginning and the end of what they think of the app.  They might not use it.  Maybe, it's not for them.  I feel like I could put it in a portfolio or take some of those App Store screens and show it to people and maybe like oh my God, this is the nicest product, you must have done a great job or your team must have worked really hard but if we built something that's really beautiful but doesn't meet the needs of our users, it's kind of I'm still superficial.  I guess part of me is still happy it's beautiful at least or that there's parts of it that are really pretty but at the end of the day as a designer, it's kind of a big fail to build something that's pretty but not the right thing.  It should be a big fail for everybody but especially as the designer, that's what you want to avoid.  06:21

\textbf{Todd:} Yeah.  So, what do you do with that tension?  06:25

\textbf{Interviewee:} During the life cycle of a product, one of the things we enable people on or if you work some more even not as a consultant, I feel like a part of your job as a designer is to constantly be educating everyone else that things that look pretty but don't work are bad designs.  I think design isn't about just making it pretty.  It's really more about making it useful.  And so, that's something that I think is a big part of our design enablement here that we work on consistently now that I'm rolling off the project.  It kind of is what it is.  I guess if I was staying on the project longer, I would keep pushing and I think the design team as a whole would keep pushing back on features that we don't think meet our user's needs and keep pushing for features that we think might not be as popular from a business perspective but would be popular with Hannah and Michael.  07:35

\textbf{Todd:} How does this affect \ldots well this is your last day.  How has this affected your motivation for the product or for the project?  07:44

\textbf{Interviewee:} Not negatively.  This is \ldots I guess I'm happy face because even though I have some question marks about some of the features, we had to prioritize and how you solve that.  There's still things we fought for that weren't on our original feature list and actually got to put in.  We did research and we know these features are going to be popular with our users and that's a huge win and also happy because I'd rather make something that looks polished or looks good even if it's not perfect functionality than something that doesn't have all perfect functionality and looks bad.  So at least if it's beautiful or elegant or it feels sexy to some users that's still your part way there.  It's just not the whole picture.  08:39

\textbf{Todd:} Cool.  So, it sounds like your motivation is either intrinsic or extrinsic.  It's outside of like I don't know is it the product set or the features or the \ldots I'm trying to ask.  Yeah, I guess what motivates you when you think about you as a designer.  08:59

\textbf{Interviewee:} I think it's more intrinsic which is why I'm happier being a designer than some of my past jobs and I think part of it is the process of building something or making something.  I really enjoy that starting with nothing and you know at the end, there's something out there that wasn't there before.  I find that kind of creative, creating and build process really satisfying.  I like the problem solving.  I think I'm motivated to solve the problems as creatively as possible or get a team to do that.  09:49

I like making things better so it's \ldots to some degree, joining a project that's a mess is hard or joining a project where people are unhappy or you know going in, it's going to be difficult.  It's hard at the same time that's where the interesting stuff is if you join a project and it's super easy and there's low expectations.  It's just not that satisfying for me because I'm not making that impact.  So I guess partly, impacting either the team around me or the product itself and the end users.  Those are probably my three things when I feel like I'm impacting all of those then I'm probably pretty motivated.  10:35

\textbf{Todd:} The team, the product and the users?  10:36

\textbf{Interviewee:} Mm-hmm.  10:37

\textbf{Todd:} Okay.  And you were describing projects that like people were unhappy or there's a mess.  Did you feel like any of that was true for this project?  10:47

\textbf{Interviewee:} The unhappiness, yeah.  10:49

\textbf{Todd:} Okay.  10:50

\textbf{Interviewee:} People have been more vocal or either people are less happy or they're more vocal about their unhappiness on this project perhaps just because it's longer and it's little different and I felt more of that coming from frankly San Francisco and I think it's because they're more process oriented.  When I worked there, it felt like they're very rigid on process and here, it's more flexible.  So, I think there just seems to be more people on this project who are feeling maybe burned out or upset about stretching or changing our process to fit Oscar.  11:34

\textbf{Todd:} Can you think of an example of where we were more flexible with our process or something we changed up maybe \ldots yeah, can you think of an example?  11:49

\textbf{Interviewee:} I mean from a design perspective, I do feel like this project more than others I've worked on here, we've had to compromise on prioritization so we're prioritizing not, it's not user-centered or design feels less user-centered than average and more business goal-centered.  So, I think there's been some tension in design around having to prioritize things that we know people don't want either are neutral or net negative for people that are going to upset users for the most part and we have to do it anyways just because there's a business reason .  And being far away from the business probably doesn't help either because instead of being able to have that one on one conversation with the decision maker, I feel like we're hearing through layers of people and so you don't feel totally heard all the time but I also know that a lot of the devs are very sensitive to that too.  I think perception-wise, it's so hard to build things.  No one wants to build something and spend all that time coding something that doesn't matter.  13:05

\textbf{Todd:} Yeah.  13:06

\textbf{Interviewee:} So, I know like when we did inception, there's a lot of questions from the dev team about things that weren't user-focused and I definitely have people commenting after, to me personally, or I overheard conversations about why we are doing this if it's not important basically or that business reason doesn't feel important to us.  13:34

\textbf{Todd:} So although the client was very focused or in part to say prioritizing around their business needs, I saw us doing a lot of user interviews so should I be surprised by that or how do I reconcile that?  13:54

\textbf{Interviewee:} So, that's a really good question and I had a conversation about this yesterday.  So, I think there's a couple gaps \ldots so partly I think it's great that they bought into that process.  Someone somewhere was ok paying that and letting us do that and that some level of buy in, that's more than maybe they had before so that's awesome.  Just because we're doing it though doesn't mean the results are being followed or respected so I do feel like we did a lot of research and then came to conclusions and people said yes, that's great, now we know that about Hannah or we know that about Michael and then things still weren't prioritized according to that.  14:44

So, you can research all you want but if the priorities don't follow the research then you're not totally buying into the process I guess.  I'd be curious to see some of the justifications where we're just not prioritizing it now.  We have those research findings, we know this about users but we're pushing back, we'll do that in a month or the next month or the next release and I think sometimes that can be really valid.  You have to prioritize and sometimes, it's not always what design wants or what a particular user wants but I hope it's not being thrown away entirely and we don't know.  15:25

\textbf{Todd:} Yeah.  15:25

\textbf{Interviewee:} The other interesting thing with research here was that while our clients were happy to let us do as much research as we could coordinate on our end, those weren't really are users.  So, the people we talked to here were American versions of Hannah and Michael but that's not actually who we're building for.  So earlier on in the process, I think there was research and we got good feedback that American Michael and German Michael were about the same, there's a lot of overlap but I think Hannah is a persona that's going to be further from a German Hannah so American Hannah and German Hannah might have a lot less overlap.  16:13

So, I am concerned that there wasn't any buy in for doing research in Germany and we asked a lot to get people to find a friend, ask a family member, email the guy that sat next to you before you came here and we couldn't get anyone to come up with even one which is a pretty sad number.  So, I think there's sort of like an outward buy in to the research but there wasn't a full buy in to the importance of the research.  16:46

\textbf{Todd:} I'd been looking at this.  So as a developer, there's a bunch of us writing code.  On iOS, there's 10-ish that's writing code in one codebase and there's this interesting tension between this idea of ownership and like \quotes{I made itπ} versus \quotes{we made it} and by looking at that dynamic and I was wondering for you as a designer, there is obviously a lot less designers working on the product, what does that look like for you?  17:16

\textbf{Interviewee:} I think it's the same type of tension.  It's really different at Pivotal which is one of the reasons I wanted to come here.  I think on my first day, I was in a kick off for a brand new project and Janice was facilitating and people were, some of our clients were asking about that some sort of proprietary and kind of around just how we do things and she said you have to be a grown up to work here, you have to be a real grown up , you have to be really, really adult to just kind of let go of a lot of the more petty things that people get hung up on in other workplaces that are okay and that just won't fly here basically.  18:01

And one of the things for me is that ego of that's my design.  I think that's super popular or super common with designers and it's sort of encouraged with talking about people like Johnny Ive who like is inventing everything at Apple.  Those are not all his designs but if you think of Apple and you think of him and people think he's kind of this God that designs all the things.  And I think some designers really, really love that the whole \quotes{that's my thingπ} and I'm not open to a ton of I do design critics but I don't really want to change my designs or I'm fine changing it but I don't want other people changing it, I don't necessarily want feedback from devs and people are kind of hung up on that.  18:46

And there's just not very much room to do that here which I think is great because it makes you, you kind of have to just if it's very Buddhist like you have to let go.  You just put it out there and then it's its own thing and you don't get to own it and it's not about who had the idea or I'm doing this and it's my baby.   You just completely have to sort of release the designs into the wild and expect it to not be yours.  And I think that's really good for me personally.  19:23

\textbf{Todd:} Why so?  19:25

\textbf{Interviewee:} Because I think it forces me to think about what's best for the product and really get in that mind-set of what's best for the team and not what's best for my special baby, this one product or this one part of the product.  And the more the larger project is, the more important that is because then you start getting like in my last company, we were building a suite of apps.  Each app had its product owner or product manager and those product managers would fight ruthlessly for resources and attention and designers and developers and things like that and it cannibalized the business vision as a whole.  So, one app was getting way more resources because their PM fought more effectively for it and other apps weren't and those other apps were just as important to kind of building a platform.  20:21

And I see it a lot even within one product if it's chunked into pages, if one product gets all the love or has a more experienced designer devs that are better with what they're doing on it and others are left to kind of languish.  It feels unbalanced within the product when everyone should be sort of working towards we want the best possible product that's best for users, it doesn't matter if I made the design or Karina made the design, I started it and she picked it up.  We just want the best thing out there.  21:01

\textbf{Todd:} Cool.  So, I think I've got it in my words hearing you've transitioned from like \quotes{I made this} to \quotes{we made this} and is that true?  21:12

\textbf{Interviewee:} Mm-hmm.  21:13

\textbf{Todd:} And I was curious from your perspective, what things enhanced or detracted from the \quotes{we made this} like what things would \ldots have you noticed that or you feel pretty bought in with the \quotes{we made this} like you feel like yes, we did make this?  21:31

\textbf{Interviewee:} There's this special magical balance that's hard to find and I don't think you can predefine it of you don't want it to be all about this my design, the designer owning it or the devs owning it but there's definitely you can swing the other way with too many cooks in the kitchen or no one taking ownership and no one feeling bought in enough to turn down and I'm having this decision or making it, someone has to make a decision and be the voice of the product if you're doing balance team or the voice of the user. And sometimes, I think we skewed towards the end of \ldots we spent too much time discussing or being sensitive with each other's designs and not just saying, okay well even though this is my idea, I still think it's best and promoting ourselves in saying my idea is great, let's go with it or I know you have all these other ideas but I think none of them are in service to the users as much as what we've got so let's not do that.  22:31

Sometimes, I think we end up being too sensitive especially the larger the team gets because we want everyone to feel included and to feel ownership but then it's like doing things by committee  then maybe the product isn't as strong or nothing, things aren't getting done as quickly.  So to be specific, we started feeling that as a design team when we had four of us working, we really liked to having, there were times when having two pairs being able to work independently was great but when all four of us sometimes kind of coalesced in one thing sometimes it was great and we had a really unexpected new idea that made the clients happy and made our users happy but other times, we felt like it was just too many people like decisions weren't being made, yeah.  23:35

\textbf{Todd:} One of the things that developers struggle with is well, we like this idea of constantly improving the codebase and with a large team, it can happen that you might say, \quotes{Oh this thing here needs improving, someone else will do it} and if everyone is doing that then nothing's going to improve.  I was curious if there was a corollary to that with design?  23:53

\textbf{Interviewee:} It's like design refactoring.  23:55

\textbf{Todd:} Yeah, I mean are there things where you're kind of purposefully make sure that the forest is cleaner at the end of the day then at the beginning?  24:03

\textbf{Interviewee:} Honestly, I feel like if you're thinking of it as like what's the grunt work that most people wish they could avoid, it's more things like we're doing now like at the end, we're doing a lot of dev support which I think everyone here loves working with devs and dev pairing but things like language support where you're like oh, I had this great design, we'd all agreed, it works and now we keep putting it into like 15 new languages and then tweaking and having to kind of solve these hard problems that are ill defined and really, really tiny.  For me, that's what I would prefer to avoid or like tracker clean up, things like that.  24:46

Occasionally, we have our InVision prototypes and at the end, it would be great if we could carve out time now to say okay let's do a prototype clean up to make sure that there is nothing in there that shouldn't be, we have every single screen we need there and it's just all clean and all the links are working.  Probably, we don't really want to do that and it might not get done.  So, there is a little bit of clean up but I think it's even less defined than refactoring for devs so.  25:24

\textbf{Todd:} We have this idea of making our tools better.  Do you have that same idea too like taking time to \ldots? 25:54

\textbf{Interviewee:} I think designers especially enthusiastic ones \ldots well designers tend to fall into two camps.  A lot of the designers around here like every time there's a new tool, we don't make our own tools as much I think as devs so every time there is a new tool, people want to test it out and sometimes it's great like \quotes{Hey, there's a thing called Sketch, let's try that} instead of using Photoshop and Illustrator which was the best thing that's happened ever.  26:01

Other times I mean since I've been in Palo Alto, there's probably been at least 10 Slack channel discussions about a new prototyping tool that people are trying and then it isn't really different or useful or worth the time of exploring.  So, you have to find that balance.  If you want to stay current, you would definitely want to update to a new tool.  If it's going to be a time saver, you're going to love it more if it works, if it's going to be better for the team but I don't think we should be trying every new tool or spending more time on updating tools and we aren't actually designing and doing dev support.  26:43

\textbf{Todd:} You were discussing a while earlier about not feeling heard by the client and I wanted to briefly explore how that affects your sense of \quotes{we built this} or your sense of ownership.  Are they related or they're not related?  27:01

\textbf{Interviewee:} Yeah, I guess in a couple of ways to one degree.  Sometimes, I've catch myself thinking I'm glad this isn't just my design because if I was the only designer and people were kind of like Andrea is the designer for this product and the product is out in the marketplace and there were a bunch of features in it that I didn't want in it that business is prioritizing and everyone looks at it as \quotes{Ooh, what designer okayed that.}  I kind of would have this inner dialogue with myself about \quotes{Oh my God are people judging me} because I designed really spammy right aloud in the product, these really spammy notifications and some that like anybody could look at and assume would probably annoy lots of people but at least on a larger product when there's many designers, I don't feel as much ownership, I don't feel as much stressed about business decision or product decisions or other designer decisions that I really disagree with.  28:12

It's kind of like well, it's not my baby, you can't win every battle and trying to be a good teammate and compromise and we have to be okay with some of the business decisions especially because I don't know everything about the business.  They could be amazingly valid business decisions and that's not my role in this project to question them or even know all about them.  So, there's things that don't might even feel wrong to me but again, it's not just my thing, I'm not the user and I'm not the only designer so it might be really right for reasons I just don't know about.  28:50

\textbf{Todd:} For us as programmers, we have this interesting schism between the code and the product and it's possible if you're really excited about the product but the code is a mess.  It's also likewise easy for us to feel great about the code but the product is a mess.  Do you see that happening?  I was just wondering if you like is \ldots well, I was hoping you would answer that question.  29:20

\textbf{Interviewee:} I think it gets back to this.  This is the most common one I see in design that I feel of \ldots 29:28

\textbf{Todd:} It's pretty but no one wants to use it.  29:31

\textbf{Interviewee:} And people get really upset and the Dribbblification of design of this, hey as long as it's beautiful or let's just copy someone else's beautiful thing without their being fought behind it or like just because it's beautiful, someone else makes something beautiful and functional and you copy the beauty and you use the same exact colors and fonts and everything make it look really slick but like maybe that doesn't work for your product or that's just copying, it's cheating.  So, I think that's the schism for us as you can look at something and appreciate the visual design but you can't even always assess I guess the user's experience or the interaction design.  Sometimes, you can but unless you know the users.  If I'm not even the intended user, it's maybe in another language.  I would never understand that market space.  I mean it's an enterprise product.  I can't really even assess if it's good or not.  It's not my business to but everyone can look at something the design and say, it's pretty.  30:41

\textbf{Todd:} So in your mind, there's this like there's the UX, how the features are used and then there's the painting, the design, the glossy looks.  30:53

\textbf{Interviewee:} There's overlap.  There's definitely an intersection and probably like with developers if the code is really great and maybe more likely to have a product that you're also very proud of, if your code is absolutely a mess it's hard to have a product that's not buggy which might bring a lot of people to, might make you feel worse about your product.  31:13

\textbf{Todd:} Yeah.  31:13

\textbf{Interviewee:} So for designers, looking pretty is part of it.  It's just sort of maybe your product version of, people can't see the code underneath so you can have code that's a mess and as long as the product is great you could say, \quotes{Hey, we're proud of the product, we just know we need to refactor.}  If it's beautiful, a lot of people look at the product and say, \quotes{Oh it's beautiful, it must be great} or they're willing to try it in a way that maybe they're not if it's not beautiful but the interaction and how users actually work with your product is a little more hidden so sometimes that's the part that beautiful product, you may just assume is good and it's not.  Alternately, you can have a phenomenal product that is simple and works for users and meets their needs really well and it's ugly and that can be okay too and people can be okay with it.  There's a lot, Google like Gmail.  32:11

\textbf{Todd:} Yeah.  32:11

\textbf{Interviewee:} Not beautiful but do most people use it and totally appreciate it?  Yes, it can be a great product and not have like a fantastic visual.  32:22

\textbf{Todd:} I was thinking about your workflow, roughly the features, the personas and the features and the filling in apps and then the sketching and like you test that and you have the concrete mocks and then a bunch of engineers actually implement it.  Does your sense of ownership like transfer at each stage or there could be gaps like I don't \ldots are there different sense of it like that's more real or I'm more invested in?  32:57

\textbf{Interviewee:} Yeah, and I don't know if most designers would feel this way or if it's really just very personal but for me, I'd feel more ownership over like wireframes and visual design because it's I think a lot more tangible and then I feel more ownership weirdly over the UI which I don't even build but I feel more empowered like the UI reflects more on me so like I'm excited to do design dev pairing because that's what I want to be adjusting pixels or I want to be making sure that the color that's being displayed is the color that we intended and if it's not, what do we do.  33:37

The research part for me, I feel less ownership over that and I don't even really know why.  I think it's just less visual and tangible like it's insights but then the whole goal is to share those and it's often a group in a room and they're not my insights, it's stuff I'm picking out of someone else's brain so I feel less ownership over the research process and even the insights than I would over the wireframes and the visual.  34:13

\textbf{Todd:} You mentioned you feel a lot of ownership of the product, the UI that the developers build.  Do you feel any ownership of the code?  34:18

\textbf{Interviewee:} No.  34:21

\textbf{Todd:} Yeah.  34:22

\textbf{Interviewee:} Not unless I've helped code.  So, I can code so when I do then I feel ownership of it.  If I built the website and it's in the code.  34:33

\textbf{Todd:} I made that css.  34:34

\textbf{Interviewee:} Yeah.  34:36

\textbf{Todd:} Okay.  How about our process?  Do you feel ownership over the way we do software development here?  34:44

\textbf{Interviewee:} I wouldn't say ownership.  I think I'm more focused on mastery or levels of understanding of the process so in facilitation of the process or enablement of it.  I don't feel that it's mine I guess.  I didn't invent it and I'm really conscious of that fact .  I think like Janice invented some things that I do or whoever came up with XP inventing, came up with the concepts that person feels like the owner to me.  I feel like I'm practitioner and I think more about like how well am I practicing somebody else's process or how long I'm like embodying the Pivotal process, how much do I agree with somebody else's process not so much ownership.  35:38

\textbf{Todd:} Do you feel empowered to change it or do you change it?  35:41

\textbf{Interviewee:} I do but I feel like I need to maser stuff before.  It's sort of like learn it and then feel free to break it.  You learn to do it the right way and then you can break it as much as you want and it's okay but I'm starting \ldots I'm just, I feel like the last couple of months, I started feeling comfortable with the idea that I want to create some of my own concepts or processes and that's okay and that's stuff that I'm starting to be comfortable teaching to other people too.  It's like if I see something enough and I believe it, I'm going to start doing it and if it works, it's okay for me to start talking about it as if it's a real thing, yeah.  36:24

\textbf{Todd:} Cool.  We've talked about a lot of different concepts here.  Tell me one more thing if there's anything percolating in there.  I'd love to hear it.  36:42

\textbf{Interviewee:} I guess I always wonder about \ldots I mean we should, I should go talk someone.  It is weird to me that I feel a lot of ownership over the end product and over the UI of it almost but no ownership of the code and to some degree, I feel like I have a super easy job because I don't build anything like I get to do some of the fun stuff.  I think about it a lot, it creates some plans like being an architect but I'm not the general contractor and I'm not a builder.  I don't really execute.  I'm just sort of on the team and I wonder \ldots I'd spent a lot of time wondering how engineers feel about that and it seems different between different people and different companies but sometimes I'm like how do engineers feel I here about designers and is that annoying to get designs then have to, do you feel like you just have to execute on someone else's ideas, are there people feeling really relieved like \quotes{thank God, I don't have to do this myself} because it's really boring or difficult.   37:52

\textbf{Todd:} Interesting.  37:54

\textbf{Interviewee:} Yeah, so I feel like there should be tension there because I feel like I got the cool job.  37:59

\textbf{Todd:} You think there should be tension between engineers and design?  38:03

\textbf{Interviewee:} I'm surprised there isn't more I guess.  38:05

\textbf{Todd:} Yeah.  38:05

\textbf{Interviewee:} I don't feel like I've felt it here in Palo Alto very much at all.  I felt it in some other jobs or companies with individuals who like to design and code and sort of and especially past jobs where everyone had more ownership over their domain, it felt more like hey you're stepping on my toes if you're trying to design and code but I'm \ldots 38:32

\textbf{Todd:} Can I answer your question from my perspective?  38:34

\textbf{Interviewee:} Okay.  38:34

\textbf{Todd:} If that's helpful.  38:35

\textbf{Interviewee:} Yeah.  38:35

\textbf{Todd:} So, I don't work a lot with you.  My first designer was Aaron, I worked a ton with and what I really appreciate about Aaron is \ldots so when I see that conflict is mostly like designer saying this is exactly what you have to build and that thing should be three pixels to the right and it's very like it's legalistic kind of perfectionism approach and Aaron had a refreshing sort of this is his style just openness and just saying \quotes{Hey, if this is not working, I want you to help me} like he viewed that the developers is almost as first line of users and having him actively listen to our input saying like Aaron we're building this and we don't think this is going to work like the UX here is not working for us and we think these are the issues and he will listen to that.  39:29

And so, that made it like really fun to work with him because it felt like there was a space that we both got to play in.  So, I don't know how every developer feels it.  For us, there's so much creativity in the code that I think losing the creativity in the mock ups is fine and I still found Aaron, a lot of the designers very open to like \quotes{Hey, I don't think that's the right font here} like I can't see those or that \quotes{Man, that really looks really close in the edge, should we add some padding} so even though I don't have a design training like showing it with some design ideas was often welcomed.  40:06

\textbf{Interviewee:} Okay.  40:07

\textbf{Todd:} As oppose to like \quotes{Oh no, this is my design, you can't change it.}  So, I think perhaps the fact that there's flexibility with the designers in the way that you build your designers and it's more of the \quotes{we} as oppose as \quotes{I built this} enables the developers to enter into the \quotes{we.}  40:23

\textbf{Interviewee:} Yeah.  40:23

\textbf{Todd:} In a way that's healthy.  40:26

\textbf{Interviewee:} And I think partly that's why there seems to be less tension here with that because we don't have that, no one has that ownership.  So, people are a lot more flexible all across the board with the changing things or accepting feedback or collaborating .  40:41

\textbf{Todd:} And then on websites, Aaron didn't know much about like CSS encoding so often, we would put him down on the IDE and he could like up and down it like Chrome whereas Richard will get in there and start messing with the CSS and so, those are kind of cool to see that like difference.  In iOS, I think it's harder for designers to come in and like it's harder for us to tweak things.  41:05

\textbf{Interviewee:} Yeah.  41:05

\textbf{Todd:} So, it's hard for me to draw conclusions from this project.  Is that helpful?  41:13

\textbf{Interviewee:} Yeah.  41:13

\textbf{Todd:} Okay.  41:14

\textbf{Interviewee:} Yeah, I think everyone is different.  I just always wonder, yeah, I don't know.  Sometimes, there's tensions and sometimes there's not and I just wonder like \ldots 41:30

\textbf{Todd:} Do you felt tension here at Pivotal?  41:32

\textbf{Interviewee:} There was some of my first projects but it was more I think the devs, it was my first, it was first project at Pivotal for me and I was the only designer and we had three PMs and that was kind of a mess and the PM that was turning, was supposed to be the full-time long term one was brand new also and then we had very experienced devs.  And so, I kept wanting feedback from him, I wanted help and I'm not.  I hadn't \ldots I'd been a non-visual designers so at my last jobs, they were very silo visual and interaction were separate so it's my first job where I was also responsible for the visual design and I really wanted that input of you're screwing up or this isn't a Pivotal way of doing things or hey, we think this would look better like this and I kept asking and they weren't giving it but then sometimes they joke about it after the fact that's like you could have just told me, I really did want you to tell me if there was 30 shades of grey, why didn't you stop me at 15.  42:42

\textbf{Todd:} Yeah.  42:44

\textbf{Interviewee:} So, there was kind of some emerging especially visual design where I was like I wish you got, you didn't to worry about stepping on my toes, I really wanted you to come voiced concerns earlier and not wait till there was dev design pairing or clean up at the end.  43:00

\textbf{Todd:} Yeah.  43:00

\textbf{Interviewee:} But I think they just didn't feel as empowered to do that like I would freak out if they did and then we got another dev who just loved front end and he was basically like I'm going to tell you if I don't like something and we're really open about it and then that was great and we didn't always agree and sometimes, we'd have to compromise or one of us will win but he was really upfront and that he would become kind of the bridge in the project to just giving really honest criticism of what was going on.  43:33

And then we had another developer who was rolled on that was really interested in research and she became sort of the bridge between design and dev just in terms of why we are doing and so she communicated a lot like hey, why are we doing what we are doing or this is why we've prioritized but not since then.  It's the only the project where it weirdly even though it was in a huge team, it did feel more siloed and partly I think that San Francisco like people are more was very like that's the process, I'm not going to \ldots44:08

\textbf{Todd:} Interesting.  44:10

\textbf{Interviewee:} So, I just felt \ldots I don't know.  I haven't felt since then like there needed to be a bridge because it's felt more like collaborative and whole team focused so I love it more or less like we needed a rep and that person was going to communicate to this people and \ldots  44:27

\textbf{Todd:} Wow.  44:27

\textbf{Interviewee:} Yeah.  44:28

\textbf{Todd:} I have appreciated Richard just asking occasionally hey what do you think of this or I've got these two things or I got these six things which ones and like that sort of pulling.  I felt like that was fun and it's nice to be involved.  It's always nice if someone asked you for your opinion and I appreciated that.  Cool.  Anything else about the topic of ownership or yeah?  44:59

\textbf{Interviewee:} I don't think so.  45:04

\textbf{Todd:} Do you have any feedback for me as an interviewer?  45:07

\textbf{Interviewee:} Here's one.  You could try and I've tried this and I don't know how successful it is but I would love if you try it and let me know; doing like mirroring of posture and stuff.  45:31

\textbf{Todd:} Okay.  45:31

\textbf{Interviewee:} Because a little bit like with you sitting back like this, at some point in the back of my mind and I didn't realize that you asked for feedback .  I felt a little more like therapist like you were like taking notes like there was a lot going on in your mind that I wasn't privy too and I was like oh I'm feeling a little more judged.  And there is like you have to do note taking and I'm wondering if yeah like sitting like this just makes me feel different about your role in our conversation.  46:06

\textbf{Todd:} Thank you.  46:06

\textbf{Interviewee:} So, I'm wondering yeah, let me know if you changed posture or something and try that out and if you feel any different about how it goes because I'm trying to figure that out for myself and I haven't come to a conclusion yet.  The fact is that I just really want to know \ldots  46:26

\textbf{Todd:} Yeah.  46:26

\textbf{Interviewee:} What's happening.  46:29

\textbf{Todd:} With what's on the paper or with my research?  46:32

\textbf{Interviewee:} Both but specifically what's on the paper.  46:35

\textbf{Todd:} It was not my intention for you to feel judged.  It's the end of a long day for me and I'm very tired.  46:41

\textbf{Interviewee:} I don't really feel judged, rationally not judged because I know like I just don't feel like you're a judgmental person.  If I didn't know you though then maybe I'm thinking maybe I would have felt a little more analyzed I guess more than judged just like analyzed.  46:57

\textbf{Todd:} I'm having a hard time keeping all my thoughts in my head and these notes were just, normally I would keep them in my head that they were reminders for me to ask you about certain things.  47:09

\textbf{Interviewee:} Okay.  47:09

\textbf{Todd:} So normally I can just follow where you're leading me but if there's a fork, I take a note about the other path and so, you might have noticed in our conversation a few times I said earlier you said or even a while ago you said, it was a while.  I can't even remember the steps so that's what these are.  47:32

\textbf{Interviewee:} The symbol stuff that like are they \ldots?  47:35

\textbf{Todd:} These are just doodles to keep me focused.  47:38

\textbf{Interviewee:} Focused, okay.  I thought it was like super special things that you were \ldots  47:46

\textbf{Todd:} That might say more about you than me.  47:48

\textbf{Interviewee:} It might.  47:49

\textbf{Todd:} But yeah, usually I don't even take notes.  47:54

\textbf{Interviewee:} Okay, cool, all right.  I was wondering if you had like \ldots I was wondering if it was a research method or you're like I don't want to rate like words because that could end up being leading or biasing so I'm going to rate symbols instead.  48:06

\textbf{Todd:} Yeah, I felt very sensitive when I led with the word ownership.  48:10

\textbf{Interviewee:} Really.  48:10

\textbf{Todd:} Because you haven't said the word ownership at all.  48:12

\textbf{Interviewee:} I didn't notice.  48:13

\textbf{Todd:} And I was like screwed, I'm leading you, I was just like and you gave me a great answer.  I really appreciated it.  48:22

\textbf{Interviewee:} Okay, it didn't feel leading to me.  48:24

\textbf{Todd:} Yeah.  48:25

\textbf{Interviewee:} So and I didn't even notice.  If you ask me now, I would have said that I brought up ownership so that works.  48:31

\textbf{Todd:} Yeah.  48:32




% \input{interviews}

\bibliographystyle{IEEEtran}
\bibliography{bibliography}

\backmatter


\end{document}